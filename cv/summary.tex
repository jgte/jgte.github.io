% !TEX root = ./summary_jgte.tex

\begin{cvtext}{Summary}

I am a researcher in the field of satellite geodesy, with a background in Aerospace Engineering.

%PhD expertise
During my PhD, I have worked with different types of satellite gravimetric data, namely \acl{hlsst} (\aclp{KO}), \acl{llsst} (\aclp{ISR}), and \acl{SGG} (differential accelerometer measurements).
My PhD research focused on:
\begin{itemize}[topsep=0pt,itemsep=1pt,parsep=0pt,partopsep=0pt]
\item modelling the data errors accurately and how its amplitude and spectra influences the quality of the resulting gravity field models;
\item quantifying the error budget of future gravimetric satellite missions, to an unprecedented level of detail;
\item analysing several mission concepts and modelled their error budget in terms of the observations and gravity field parameters;
\item establishing that a constellation of numerous non-dedicated satellites make it possible to measure fast mass transport processes;
\item demonstrating that some mission concepts (those with large radial distances, \acs{e.g.} the cartwheel formation) are very sensitive to particular types of errors (specifically positioning errors); %
\item proved that alternative mission concepts (the cross-track pendulum formation) are much better suited to complement planned future gravimetric missions.%
\end{itemize}
This has allowed me to study future gravimetric missions in detail, even unconventional ones such as augmenting dedicated gravimetric missions with a large constellation of non-dedicated satellites.
My expertise on this topic has been noted by peers, who have invited me to participate in numerous research projects involving international teams.

%before CSR
As a Post-doctoral Fellow at \ac{TUD}, I dedicated my efforts to implement the Level 2 data processing facility of the \dynhref{https://earth.esa.int/web/guest/missions/esa-operational-eo-missions/swarm}{Swarm satellite mission}, concerning the Precise Orbit Determination and Thermospheric Neutral Density processing streams.
I have acquired expertise in \ac{DSP} techniques and contributed to the processing of Swarm accelerometer data, by combining non-gravitational accelerations derived from \ac{GPS} data and the accelerometer measurements.
In doing so, I have gratly removed the long-term bias in the accelerometer data.
During this time, I also matured my skills in data management and automated processing.

%current situation
Currently as a Post-doctoral Fellow at \ac{CSR} of the \ac{UTexas}, I am improving the calibration of \ac{GRACE} accelerometer data, which is particular relevant after 2011, when the thermal control on the satellites was switched off.
I am also developing time-series analysis techniques to predict the long-term trends in the \ac{GRACE} gravity field models over the \ac{GRACE}/\ac{GRACE-FO} gap.
This research will ensure the measurements collected by the two missions are not biased and the mass estimates are continuous.

%cooperation
I have developed a wide and strong network (AT, CH, CZ, DE, NL, PT and US).
I took the lead in coordinating and with several European and US institutes the research and promotion the gravity field models estimated from the \ac{GPS} data gathered by the Swarm satellite mission.
I also successfully applied for funding to the \ac{DISC} consortium, allowing these activities to proceed smoothly.
In an effort to promote the use of nano-satellites for collecting gravimetric data, I am currently cooperating with \ac{UMinho} and CSR to develop a \ac{MEMS}-based micro accelerometer and investigate practical CubeSat architectures.
The objective of this work is to demonstrate the feasibility of a CubeSat to vastly increase the spatial and temporal sampling of the collected gravimetric data.

%background expertise
I have studied and worked in numerous areas, including Structural Mechanics, Aerodynamics, Preliminary Vehicle Design, Single Stage to Orbit and Laser Propulsion, which have given me the opportunity to broaden my understanding of Physics.
I am an avid programmer, actively learning new languages and techniques in order to better implement the algorithms and procedures required to develop my research.
I openly share the code I developed in \dynhref{http://github.com/jgte}{GitHub}.

\end{cvtext}