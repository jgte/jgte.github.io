\documentclass[a4paper]{article}

% ------ packages --------------------------------------------------------

% sudo tlmgr install titlesec biblatex logreq biber tex4ht

\usepackage{hyperref,etoolbox,supertabular,multirow}

\usepackage{titlesec}% Used to customize the \section command
\titleformat{\section}{\sf\Large\raggedright}{}{0em}{}[\titlerule] % Text formatting of sections
\titlespacing{\section}{0pt}{3pt}{3pt} % Spacing around sections
\titleformat{\subsection}{\sf\large\raggedright}{}{0em}{}[] % Text formatting of sections
\titlespacing{\subsection}{0pt}{3pt}{3pt} % Spacing around sections

% \usepackage{natbib}
\usepackage[sorting=ydnt,backend=biber,style=authoryear,maxnames=99]{biblatex}
\addbibresource{cv_jgte.bib}

\usepackage[utf8]{inputenc}    % utf8 support       %!!!!!!!!!!!!!!!!!!!!
\DeclareUnicodeCharacter{301}{'}

\usepackage[T1]{fontenc}       % code for pdf file  %!!!!!!!!!!!!!!!!!!!!
\usepackage{longtable}
\usepackage{fancyhdr}
\usepackage{tabularx}

% ------ Name --------------------------------------------------------

% Set your name here
\def\name{Jo\~ao~de~Teixeira~da~Encarna\c c\~ao}


% Replace this with a link to your CV if you like, or set it empty
% (as in \def\footerlink{}) to remove the link in the footer:
\def\footerlink{}

% The following metadata will show up in the PDF properties
\hypersetup{
  colorlinks = true,
  urlcolor = blue,
  pdfauthor = {\name},
  pdfkeywords = {},
  pdftitle = {\name: Curriculum Vitae},
  pdfsubject = {Curriculum Vitae},
  pdfpagemode = UseNone
}

\author{\name}

% ------ dynamic web references -------------------------------

\newtoggle{expliciturl}
\togglefalse{expliciturl}
% \toggletrue{expliciturl}
\newcommand{\dynhref}[2]{%
  \iftoggle{expliciturl}{%
    #2 (\href{#1}{\texttt{\detokenize{#1}}})%
  }{%
    \href{#1}{#2}%
  }%
}

% ------ CV version: professional or academic -------------------------------

\newtoggle{homecontact}
\togglefalse{homecontact}
% \toggletrue{homecontact}
\newcommand{\homecv}[2]{\iftoggle{homecontact}{#1}{#2}}


\newtoggle{professionalcv}
\togglefalse{professionalcv}
% \toggletrue{professionalcv}
\newcommand{\procv}[2]{\iftoggle{professionalcv}{#1}{#2}}

% ------ Margin control --------------------------------------------------------

\addtolength{\oddsidemargin}{-1.5cm}
\addtolength{\evensidemargin}{-1.5cm}
\addtolength{\textwidth}{3cm}

\addtolength{\topmargin}{-1.5cm}
\addtolength{\textheight}{2.5cm}

% ------ Customizations --------------------------------------------------------

\pagestyle{fancy}
\fancyhf{}
\rhead{\it{\name}}
\lhead{}
\rfoot{\thepage}
\thispagestyle{empty}


% % Customize page headers
% \pagestyle{myheadings}
% \markright{\it{\name}}

% Don't indent paragraphs.
\setlength\parindent{0em}

% Make lists without bullets
\renewenvironment{itemize}{
  \begin{list}{}{
    \setlength{\leftmargin}{1.5em}
  }
}{
  \end{list}
}
\newlength{\listskipbig}
 \setlength\listskipbig{0.15cm}
\newlength{\listskipsmall}
 \setlength\listskipsmall{0.1cm}


\newenvironment{cvsection}[2]{
  \vspace{0.15in}
  \section*{#1}
  \vspace{-0.2in}
  % \def\arraystretch{1.5}
  \begin{longtable}{lp{#2}}
  % \begin{tabularx}{\textwidth}{lX}
  % \begin{tabular*}{\textwidth}{ll}
}{
  % \end{tabular*}
  % \end{tabularx}
  \end{longtable}
}

\newenvironment{cvtext}[1]{
  \vspace{0.15in}
  \section*{#1}
  \begin{minipage}{\textwidth}
  \setlength{\parindent}{10ex}
  \raggedright
}{
  % \end{tabular*}
  % \end{tabularx}
  \end{minipage}
}
% ------ Start --------------------------------------------------------

\begin{document}

% print name centered and bold:
\centerline
{\huge \rm \textbf \name}

\procv{}{%
\vspace{0.25in}
\centering
Postdoctoral Fellow, Center for Space Research, University of Texas at Austin
}

\begin{cvtext}{Summary}
\hspace{3ex} Jo\~ao Encarna\c c\~ao is a researcher in the field of satellite geodesy.
He has worked with different types of gravimetric data, focusing on understanding their error characteristics and how that influences the quality of the resulting gravity field models.
He participated in numerous research projects involving international teams, which has allowed him to develop a wide and strong network (AT, CH, CZ, DE, NL, PT and US).

\hspace{3ex} As a Postdoctoral Fellow at Center for Space Research, he is currently looking at ways to improve the calibration of GRACE accelerometer data and to predict the long-term trends in the GRACE gravity field models over the GRACE/GRACE Follow-on gap.
Additionally, Jo\~ao Encarna\c c\~ao leads in informal cooperation between several European institutes for researching and promoting the gravity field models estimated from the GPS data gathered by the Swarm satellite mission.

\hspace{3ex} He has worked in different areas, including Structural Mechanics, Aerodynamics, Preliminary Vehicle Design, Single Stage to Orbit and Laser Propulsion, which have given him the opportunity to broaden his understanding of physics.
Jo\~ao Encarna\c c\~ao is an avid programmer, actively learning new languages and techniques in order to better implement the algorithms and procedures required to develop his research. He openly shares the code he has developed in \dynhref{https://github.com/jgte}{GitHub}.

\hspace{3ex} An \dynhref{http://jgte.github.io/cv/}{on-line version this document} is also available.

\end{cvtext}

% ------- Personal data ---------------------------------------------------

\begin{cvsection}{Personal Information}{14cm}

{\bf Full Name:} & Jo\~ao~Greg\'orio~de~Teixeira~da~Encarna\c c\~ao \\[\listskipsmall]
{\bf Birth:} & $25^{th}$ of February 1977 at Funchal, Portugal \\[\listskipsmall]
{\bf Nationality:} &  Portuguese\\[\listskipsmall]

\homecv{%
  {\bf Marital Status:} & Single \\[\listskipsmall]
  {\bf Address:}   & 4303 Duval Street 302 \newline
                    78751, Austin Texas, USA\\[\listskipsmall]
  {\bf Telephone:} & +1 512 765 1351\\[\listskipsmall]
  {\bf Email:}     & \href{mailto:j_encarnacao@yahoo.com}{j\_encarnacao@yahoo.com}\\[\listskipsmall]
  {\bf Web:}       & \dynhref{http://nl.linkedin.com/in/joaoencarnacao}{LinkedIn}\\[\listskipsmall]
}{%
  {\bf Address:}   & 3925 W Braker Lane Ste 200 - WPR 2.9076\newline
                     Austin TX 78759-5316\newline
                    USA\\[\listskipsmall]
  {\bf Telephone:} & +1 (512) 232-6897\\[\listskipsmall]
  {\bf Email:}     & \href{mailto:teixeira@csr.utexas.edu}{teixeira@csr.utexas.edu}\\[\listskipsmall]
  {\bf Web:}       & \dynhref{https://directory.utexas.edu/index.php?q=joao+encarnacao}{University of Texas}\newline
                     \dynhref{http://nl.linkedin.com/in/joaoencarnacao}{LinkedIn}\newline
                     \dynhref{https://www.researchgate.net/profile/Joao_Encarnacao2}{ResearchGate}\newline
                     \dynhref{https://scholar.google.com/citations?user=k2liFwQAAAAJ}{Google Scholar}\newline
                     \dynhref{http://orcid.org/0000-0001-6824-2733}{ORCID}\newline
                     \dynhref{https://www.mendeley.com/profiles/joao-encarnacao4/}{Mendeley}\newline
                     \dynhref{https://www.scopus.com/authid/detail.uri?authorId=15135565900}{SCOPUS}\newline
                     \dynhref{https://publons.com/a/782170/}{Publons}\newline
                     \dynhref{https://github.com/jgte}{GitHub}\\[\listskipsmall]
}
\end{cvsection}

% % ------- Skills ------------------------------------------------------

\clearpage

\begin{cvsection}{Skills}{11.2cm}
Leadership: & $\bullet$ Established and manages the cooperation between four European institutes:\newline
  \dynhref{http://www.itsg.tugraz.at}{Institute of Geodesy} of the \dynhref{http://www.tugraz.at}{Graz University of Technology}, \newline
  \dynhref{http://www.asu.cas.cz/en}{Astronomical Institute} of the \dynhref{http://www.cas.cz/index.html}{Academy of Sciences of the Czech Republic}, \newline
  \dynhref{http://www.aiub.unibe.ch}{Astronomical Institute of the University of Bern} \newline
  \dynhref{http://www.lr.tudelft.nl}{Aerospace Faculty} of the \dynhref{http://www.tudelft.nl}{Delft University of Technology} and\newline
  \dynhref{https://earthsciences.osu.edu}{School of Earth Sciences} of the \dynhref{https://www.osu.edu}{Ohio State University} \newline
  for the {\bf study of the time-variable gravity field of the Earth retrieved from GPS data from the Swarm Satellite mission}, leading to a previous grant application with very positive reviews (Ref. ESA AO/1-7927/14/NL/MP), and response to the \dynhref{http://tinyurl.com/SwarmGrav}{ITT posted by the ESA-funded DISC consortium} (on-going)\\[\listskipsmall]

            & $\bullet$ Co-initiated the \dynhref{http://doptrack.tudelft.nl}{DopTrack project}
              consisting of a {\bf satellite tracking radio station that exploits the Doppler effect}; promoted the project, secured departmental funding, selected and assembled the hardware, developed software, engaged students and mentored practical undergraduate projects.\\[\listskipbig]

Communication:  &
                  $\bullet$ \dynhref{http://tinyurl.com/h928s3c}{Invited talk} at the \dynhref{http://fallmeeting.agu.org/2015/}{American Geophysical Union Fall Meeting in 2015}\newline
                  $\bullet$ Invited lecture at the \dynhref{http://data-assimilation.com}{Summer School on Data Assimilation and its applications in Oceanography, Hydrology, Risk \& Safety and Reservoir Engineering in 2017}\newline
                  $\bullet$ Numerous presentations of research results (8 oral and 4 poster)\\[\listskipbig]

Teaching:      & $\bullet$ Student supervision in the context of individual and group assignments\newline
                 $\bullet$ Introductory lectures to the practical projects\\[\listskipbig]

Theoretical:   & $\bullet$ Parametric inversion\newline
                 $\bullet$ Statistical analysis\newline
                 $\bullet$ Stochastic modelling\newline
                 $\bullet$ Spherical harmonic functions\newline
                 $\bullet$ Digital signal processing\newline
                 $\bullet$ Coordinate transformations\slash quaternion arithmetic\newline
                 $\bullet$ Fourier analysis\\[\listskipbig]

Computational:  & $\bullet$ Algorithm development and implementation\newline
                  $\bullet$ Data management, analysis and visualisation\newline
                  $\bullet$ Automation, robustness, fault recovery\newline
                  $\bullet$ Problem resolution\slash solution discovery\slash hacking\\[\listskipbig]

Software: & Latex, MS Office, Git, SVN\\[\listskipbig]
\procv{}{
Articles review: & Successfully completed the review of 8 scientific articles cf. \dynhref{https://publons.com/a/782170/}{Publons}\\[\listskipbig]
}

Operating Systems: & OSX, MS Windows, Unix/Linux \\[\listskipbig]

Programming:
& 1996 -- present: Bash \\
& 1998 -- present: MATLAB \\
& 2002 -- present: FORTRAN \\
& 2006 -- 2008: C/C++ \\
& 2011 -- present: Ruby \\
& 2015 -- present: Python\\[\listskipbig]
\end{cvsection}

% % ------- Fields of interest ------------------------------------------

\clearpage

\begin{cvsection}{Fields of Interest}{10cm}
 Big data & \\
 Geophysics & \\
 Digital signal processing & \\
 Numerical Simulation & \\
 Rocket Motion and Orbital Mechanics & \\
 Preliminary Vehicle Design & \\
 Aerodynamics & \\
 Structural Mechanics & \\
\end{cvsection}

% % ------- Education ---------------------------------------------------

\begin{cvsection}{Education}{13.5cm}

2015 & PhD in Space Geodesy\newline
      \dynhref{http://tinyurl.com/GRS-TUDelft}{Geoscience \& Remote Sensing}, \dynhref{http://www.tudelft.nl/}{Delft University of Technology}\newline
      Dissertation: \dynhref{http://tinyurl.com/SatGrav}{Next-generation satellite gravimetry for measuring mass transport in the Earth system}\newline
      Promotor: \dynhref{http://tinyurl.com/ProfKlees}{Prof. Dr.-Ing. habil. Roland Klees}\newline
      Supervisor: \dynhref{http://tinyurl.com/DrDitmar}{dr.ir. Pavel Ditmar}\\[\listskipbig]

2004 & Master of Sciences in Aerospace Engineering\newline
      \dynhref{http://www.as.lr.tudelft.nl}{Astrodynamics and Space Missions}, \dynhref{http://www.tudelft.nl/}{Delft University of Technology}\newline
      Final Thesis: Numerical Simulation of Launch Vehicles\newline
      Supervisor: \dynhref{http://tinyurl.com/ProfAmbrosius}{Prof.ir. B.A.C. Ambrosius}\\[\listskipbig]

2000 & Licenciatura in Aerospace Engineering\newline
       \dynhref{http://www.ist.utl.pt/}{Instituto Superior T{\'e}cnico}, \dynhref{http://www.utl.pt/}{Technical University of Lisbon}\newline
       $5^{th}$ year concluded at Delft University of Technology, through the ERASMUS program\newline
       Report: Optimum Aerodynamic Shape for a High Altitude Long Endurance Aerostatic Platform\newline
       Supervisor: Prof. Dr. Ir. Theo van Holten\\[\listskipbig]

\end{cvsection}

% \citem{Scholarships}\\
% Socrates/ERASMUS Scholarship of the European Union (September 1999 -- June
% 2000)


% % -------- Work experience --------------------------------------------

\begin{cvsection}{Work Experience}{10.8cm}

Aug. 2016 -- present
  & Research Associate\newline
    \dynhref{http://www.csr.utexas.edu}{Center for Space Research}, \dynhref{http://www.utexas.edu}{Texas University at Austin}\newline
    Austin, Texas, USA\\[\listskipbig]

Sep. 2011 -- Jul. 2016
  & Research Associate\newline
    \dynhref{http://www.as.lr.tudelft.nl}{Astrodynamics and Space Missions}, \dynhref{http://www.tudelft.nl/}{Delft University of Technology}\newline
    Delft, the Netherlands\\[\listskipbig]

Jan. 2007 -- Dec. 2015
  & PhD Candidate\newline
    \dynhref{http://tinyurl.com/GRS-TUDelft}{Geoscience \& Remote Sensing}, \dynhref{http://www.tudelft.nl/}{Delft University of Technology}\newline
    Delft, the Netherlands\\[\listskipbig]

Apr. 2005 -- Nov. 2006
  & Stress Engineer\newline
    \dynhref{http://www.globaltechnics.nl/}{Global Technics}\newline
    Leiden, the Netherlands\\[\listskipbig]

% 6/2005 -- 7/2005 & Systems Engineer \newline
%                  \dynhref{http://www.reduct.net}{Reduct}\newline
%                    Kontich, Belgium\\[\listskipbig]

Oct. 2004 -- Jan. 2005
  & Aerospace Engineer\newline
  Implementation of a Sub-Orbital Optimization Module into the Simulation Tool Colvet (Contractor Work)\newline
  \dynhref{http://www.delta-utec.com/}{Delta-Utec}\newline
  Leiden, the Netherlands\\[\listskipbig]

Mar. 2004 -- Apr. 2004
  & Trainee\newline
    Numerical Simulations on Laser Propulsion (appendix of MSc thesis)\newline
    Under the supervision of Ir. Berry Sanders, Rocket Technology Research Group\newline
    Prins Maurits Laboratorium, \dynhref{http://www.tno.nl/}{TNO}, the Netherlands\\[\listskipbig]

Sep. 2001 -- Dec. 2001
  & Trainee\newline
    Lunar Exploration with Ariane 5\newline
    Under the supervision of \dynhref{https://en.wikipedia.org/wiki/Wubbo_Ockels}{Prof. Wubbo Ockels}\newline
    \dynhref{http://www.esa.int/About_Us/ESTEC}{European Space research and Technology Center (ESTEC)}, \dynhref{http://www.esa.int}{European Space Agency (ESA)}
    Noordwijk, the Netherlands\\[\listskipbig]

\end{cvsection}

% % ------- Research ------------------------------------------

\procv{}{
\begin{cvsection}{Research}{11cm}
Satellite accelerometry & $\bullet$ Combination of orbit-derived non-gravitational accelerations with accelerometer observations in the context of the Swarm satellite mission\\[\listskipbig]
Space Geodesy & $\bullet$ Gravity field models from Swarm kinematic orbits\newline
                $\bullet$ Impact of orbit position errors in the quality of gravimetric data from satellite formations\newline
                $\bullet$ Noise budget of low-low satellite-to-satellite tracking gravimetric data\newline
                $\bullet$ Retrieval of the high-frequency time-variable gravity field of the Earth with numerous satellites\\[\listskipbig]
Laser Propulsion & $\bullet$ Use of ground-based lasers to launch small satellites to orbit\\[\listskipbig]
Single Stage to Orbit & $\bullet$ Determination of the minimum technological requirements for a single stage to orbit launcher\\[\listskipbig]
\end{cvsection}
}

% % -------- Research Projects ----------------------------------------------------

\procv{}{
\begin{cvsection}{Research Projects}{12.2cm}
2013 -- 2015    & Assessment of Satellite Constellations for Monitoring the Variations in Earth's Gravity Field (ESA contract 4000108663/13/NL/MV) \\[\listskipbig]
2013          & GOCE+ Theme3: Air density and wind retrieval using GOCE data (ESA contract 400010284/11/NL/EL)\\[\listskipbig]
2011 -- 2016  & Development of the Swarm Level 2 Algorithms and Associated Level 2 Processing Facility (ESA Contract 4000102140/10/NL/JA)\\[\listskipbig]
2010          & Assessment of a Next Generation Gravity Mission for Monitoring the Variations of Earth's Gravity Field (ESTEC contract 22643/09/NL/AF)\\[\listskipbig]
2008          & Monitoring and Modelling Individual Sources of Mass Distribution and Transport in the Earth System by Means of Satellites (ESA contract 20403) \\[\listskipbig]
\end{cvsection}
}

% % ------- Conferences

% 2001 IAF (Houton) - presentation
% 2007 INTERGEO (Leipzig) - presentation
% 2008 IUGG (Chania) - poster
% 2013 1st Swarm Data Quality Workshop (Rome) - presentation
% 2013 2nd Swarm Data Quality Workshop (Rome) - presentation
% 2014 3rd Swarm Data Quality Workshop and 3rd Swarm Science Meeting (Copenhagen) - presentation
% 2014 4th Swarm Data Quality Workshop (Potsdam) - presentation
% 2015 5th Swarm Data Quality Workshop (Paris) - presentation
% 2015 AGU - presentation and poster
% 2016 EGU - poster
% 2016 LPS - poster
% 2016 AGU (no presentation)

% % ------- Languages ------------------------------------------------------

\begin{cvsection}{Languages}{9cm}
~\\
\begin{minipage}{\textwidth}
\begin{tabular}{
l
>{\centering\arraybackslash}m{3cm}
>{\centering\arraybackslash}m{3cm}
>{\centering\arraybackslash}m{3cm}}
   & Speaking & Reading & Writing \\
\hline
Portuguese & \multicolumn{3}{c}{mother tongue}  \\
English\footnote{holder of the \dynhref{http://www.cambridgeenglish.org/exams/proficiency/index.aspx}{Certificate of Proficiency in English}}    & excellent & excellent & excellent \\
Spanish    & good & good & fair \\
Italian    & good & good & fair \\
Dutch      & fair & fair & limited \\
French     & fair & fair & limited \\
\hline
\end{tabular}
\end{minipage}
\end{cvsection}

% % -------- Other Activities --------------------------------------------

\procv{
\begin{cvsection}{Sports}{8cm}
1991 -- 2009 & Basketball \\
April 2006   & Finalist of the \dynhref{http://www.fortismarathonrotterdam.nl/}{26th International Fortis Marathon of Rotterdam}
September 2016 - present & Sailing\\
\end{cvsection}

\begin{cvsection}{Other Activities}{8cm}
1991 -- 2001 & Scout at the 92$^{nd}$ Scout-group of the \dynhref{http://www.aep.pt}{Association of Portuguese Escoteiros (AEP)}\\
1993 -- present & Radio Amateur, call sign CT3IU, class B\\
1996 -- present & Drivers Licence \\
\end{cvsection}
}{}

% % -------- Publications ----------------------------------------------------

\clearpage

\procv{}{
\renewcommand\refname{Publications}
\vspace{0.25in}

% grep \@ citations.bib | awk -F'{' '{ print $2}'
\nocite{
gunter2011using,
resendes2007laser,
resendes2006laser,
hashemi2010contribution,
gunter2010using,
ditmar2010mitigation,
% mendoncca11,
encarnaccao2009influence,
ditmar2012understanding,
farahani2013validation,
gunter2009determination,
gunter2011investigation,
olsen2013swarm,
visser2013thermospheric,
doornbos2012thermospheric,
resendes2005laser,
olsen2011scarf,
gunter2012potential,
teixeira2002single,
gunter2012deriving,
anselmi2011assessment,
gunter2009use,
encarnacao2008analysis,
van2015precise,
bruinsma2014results,
doornbos2013air,
friis2013preface,
bezdvek2016time,
encarnaccao2015first,
encarnacao2015impact,
siemes2016swarm,
de2016gravity,
da2016gravity,
}

\printbibliography

~\\
Retrived from \dynhref{https://scholar.google.com/citations?user=k2liFwQAAAAJ}{Google Scholar}.
}


\end{document}
