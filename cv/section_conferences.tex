% !TEX root = ./jgte_cv.tex


\cvheading{Conference activity}
\vspace{1em}

%--------------------------------------

%NOTICE: entries with * indicate i was the presenter

\cvsubheading{Oral presentations}
\begin{refsection}
\nocite{
Encarnacao2002,              % Single Stage To Orbit Minimum Requirements Through Numerical Simulation COSPAR 2002  1*
Encarnacao2008,              % Analysis of Satellite Formations in the Context of Gravity Field        3rd ISFFMT   2*
TeixeiraEncarnacao2014,      % POD-assisted calibration of Swarms Accelerometer Data                   4th SwarmDWQ 3*
TeixeiraEncarnacao2014a,     % Combination of Swarm's Uncalibrated Accelerometer Data with POD-Based   3rd SwarmSM  4*
TeixeiraEncarnacao2014b,     % Preliminary analysis of accelerometer data                              2nd SwarmDWQ 5*
TeixeiraEncarnacao2015b,     % Frequency domain combination of POD-driven and measured accelerations   5th SwarmDWQ 6*
TeixeiraEncarnacao2015a,     % First monthly gravity field solutions derived from GPS orbits of Swarm  AGU 2015     7*
TeixeiraEncarnacao2017,      % Swarm as an Observing Platform for Large Surface Mass Transport         4th SwarmSM
Encarnacao2018,              % Signal contents of combined monthly gravity field models derived from   EGU 2018     8*
Jaggi2018a,                  % Assessment of individual and combined gravity field solutions from      EGU 2018
Visser2019,                  % Complete 5-years time series of combined monthly gravity field models   EGU 2019
Visser2018a,                 % Multi-approach Gravity Field Models from Swarm GPS data                 COSPAR 2018
Encarnacao2019c,             % Multi-approach gravity field models from Swarm GPS data                 9th SwarmDQW
Encarnacao2020,              % Multi-approach gravity field models from Swarm GPS data                 10th Swarm DQW
Encarnacao2021,              % Seven years of monthly low-degree gravity field models from Swarm ...   EGU 2021             1
Encarnacao2021a,             % Requirements and user needs in the context of quantum gravimetric ...   COM-ESA Workshop QSG 2
TeixeiradaEncarnacao2021b,   % Multi-approach gravity field models from Swarm GPS data                 11th Swarm DQW       3
Encarnacao2022,              % Eight years of temporal gravity changes observed by the Swarm ...       EGU 2022             4
Encarnacao2022a,             % Multi-approach gravity field models from Swarm GPS data                 12th Swarm DQW       5
}
\printbibliography
\end{refsection}

\cvsubheading{Poster presentations}
\begin{refsection}
\nocite{
TeixeiraEncarnacao2007,      % Temporal aliasing in GRACE monthly solutions                            INTERGEO 2007 1*
TeixeiraEncarnacao2007a,     % Influence of hydrology-related temporal aliasing on the quality of      VMSG 2007     2*
TeixeiraEncarnacao2008,      % Spectral analysis of positioning modelling errors in gravimetric data   IAG 2008      3*
Encarnacao2015,              % Impact of Orbit Position Errors on Future Satellite Gravity Models      AGU 2015      4*
TeixeiraEncarnacao2016b,     % Gravity field models derived from Swarm GPS data                        EGU 2016      5*
TeixeiraEncarnacao2017a,     % Gravity field models derived from Swarm GPS data                        EGU 2017      6*
Encarnacao2017,              % Temperature corrected-calibration of GRACE’s accelerometer              AGU 2017      7*
TeixeiradaEncarnacao2018d,   % Analysis of GRACE's accelerometer scale-factor calibration              AGU 2018      8*
DeTeixeiradaEncarnacao2018c, % GRACE’s accelerometer scale-factor calibration                          GRACE STM 2018
Zehentner2018,               % Investigations of GNSS-derived baselines for gravity field recovery     EGU 2018
DeTeixeiradaEncarnacao2018b, % Observing Earth's mass transport processes with the Swarm satellites    GGHS 2018
Zhang2018a,                  % Swarm Temporal Gravity Field Estimates Using Acceleration Approach      9th Int W on TibXS
Encarnacao2019a,             % Earth's Mass Transp. Pro. Obs. by the Swarm Sat. during the G/G-FO Gap  ESA LPS 2019
TeixeiradaEncarnacao2021a,   % Monthly low-degree gravity field models from Swarm GPS data for ...     IAG 2021     1
Encarnacao2021b,             % uPGRADE - Miniaturized Prototype for Gravity field Assessment ...       UTA-P 2021   2
TeixeiradaEncarnacao2022,    % Earth's Mass Transport Processes Observed by the Swarm Satellites ...   ESA LPS 2022 3
Encarnacao2023,              % Temporal gravity changes over nine years as observed by the Swarm ...   EGU 2023     4
Encarnacao2023a,             % Temporal gravity changes over nine years as observed by the Swarm ...   IUGG 2023    5
gunter2010using,
ditmar2010mitigation,
hashemi2010contribution,
gunter2011investigation,
gunter2012potential,
doornbos2012thermospheric,
Olsen2013,
doornbos2013air,
bruinsma2014results,
iran2014search,
astafyeva2015ionospheric,
doornbos2015processing,
jaggi2016european,
sneew2016esa,
siemes2016improvements,
doornbos2016thermospheric,
}
\printbibliography
\end{refsection}






