% !TEX root = ./jgte_cv.tex


\cvheading{Conferences}
\vspace{1em}

%--------------------------------------

\cvsubheading{Attendance}
\begin{refsection}
\nocite{
Encarnacao2002,           % Single Stage To Orbit Minimum Requirements Through Numerical Simulation COSPAR 2002   oral      3 
TeixeiraEncarnacao2007,   % Temporal aliasing in GRACE monthly solutions                            INTERGEO 2007 poster    1
TeixeiraEncarnacao2007a,  % Influence of hydrology-related temporal aliasing on the quality of      VMSG 2007     poster    2
Encarnacao2008,           % Analysis of Satellite Formations in the Context of Gravity Field        3rd ISFFMT    oral      4  
TeixeiraEncarnacao2008,   % Spectral analysis of positioning modelling errors in gravimetric data   IAG 2008      poster    3
TeixeiraEncarnacao2014,   % POD-assisted calibration of Swarms Accelerometer Data                   4th SwarmDQW  oral      5
TeixeiraEncarnacao2014a,  % Combination of Swarm's Uncalibrated Accelerometer Data with POD-Based   3rd SwarmSM   oral      6
TeixeiraEncarnacao2014b,  % Preliminary analysis of accelerometer data                              2nd SwarmDQW  oral      7
Encarnacao2015,           % Impact of Orbit Position Errors on Future Satellite Gravity Models      AGU 2015      poster    4
TeixeiraEncarnacao2015b,  % Frequency domain combination of POD-driven and measured accelerations   5th SwarmDQW  oral      8
TeixeiraEncarnacao2016b,  % Gravity field models derived from Swarm GPS data                        EGU 2016      poster    5
TeixeiraEncarnacao2017a,  % Gravity field models derived from Swarm GPS data                        EGU 2017      poster    6
Encarnacao2017,           % Temperature corrected-calibration of GRACE’s accelerometer              AGU 2017      poster    7
TeixeiradaEncarnacao2018d,% Analysis of GRACE's accelerometer scale-factor calibration              AGU 2018      poster    8
}
\printbibliography
\end{refsection}

\cvsubheading{Contributions}
\begin{refsection}
\nocite{
Visser2019,                  %Complete 5-years time series of combined monthly gravity field models derived from Swarm GPS data
DeTeixeiradaEncarnacao2018c, %GRACE’s accelerometer scale-factor calibration
Encarnacao2018,              %Signal contents of combined monthly gravity field models derived from Swarm GPS data, EGU
Zehentner2018,               %Investigations of GNSS-derived baselines for gravity field recovery, EGU
Jaggi2018a,                  %Assessment of individual and combined gravity field solutions from Swarm GPS data and mitigation of systematic errors, EGU
TeixeiraEncarnacao2017,      %Swarm as an Observing Platform for Large Surface Mass Transport Processes Banff 2017
DeTeixeiradaEncarnacao2018b, % Observing Earth's mass transport processes with the Swarm satellites, GGHS-2018
Zhang2018,                   % Swarm Temporal Gravity Field Estimates Using Acceleration Approach, 9th International Workshop on TibXS
Visser2018a,                 % Multi-approach Gravity Field Models from Swarm GPS data, COSPAR
gunter2010using,
ditmar2010mitigation,
hashemi2010contribution,
gunter2011investigation,
gunter2012potential,
doornbos2012thermospheric,
Olsen2013,
doornbos2013air,
bruinsma2014results,
iran2014search,
astafyeva2015ionospheric,
doornbos2015processing,
Siemes2015,
jaggi2016european,
sneew2016esa,
siemes2016improvements,
doornbos2016thermospheric,
}
\printbibliography
\end{refsection}






