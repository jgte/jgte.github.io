% !TEX root = ./jgte_publication_summary.tex

\begin{cvtext}{Summary of selected publications}

In \cite{TeixeiraEncarnacao2016}, the habitability of the Swarm satellites to observe changes on Earth's gravity field is for the first time demonstrated.
This research was one in cooperation with five international research institutes (
\acf{IfG} of the \acf{TUG},
\ac{ASU}  of the \acf{AVCR},
\acf{AIUB},
\dynhref{www.lr.tudelft.nl}{Aerospace Faculty} of the \ac{TUD} and
\acf{SES} of the \acf{OSU}), 
within the context of a project initiated and lead by me.
Since publishing this article, I have taken the leading role securing funding from the \ac{ESA}, through the \ac{DISC}, to develop the methodology to best combine the individual gravity field solutions from each research institute. This project has recently been awarded additional funding to routinely produce monthly models for the foreseeable future.
The Swarm gravity field models constitute an independent source of gravimetric data from dedicated mission (\ac{GRACE} and \ac{GRACE-FO}).
Although their spatial resolution is much lower, the Swarm models are still able to describe monthly variations of Earth's gravity field within a few millimetres geoid height, when compared with \ac{GRACE}.
This has proven to be particularly beneficial in monitoring mass transport processes during the gap between the periods when \ac{GRACE} and \ac{GRACE-FO} were and are (respectively) collecting data.

My contribution to \cite{Ditmar2012} demonstrates that the influence of temporal aliasing caused by errors in the \ac{AOD} models, needed to remove rapid-changing gravity variations that are impossible for \ac{GRACE} to observe, cannot explain the unexpected low quality of the gravity field models, compared with pre-launch predictions, particularly in what concerns the striping artefacts.
Additionally, the results of this publication contrast with other similar studies in the sense that it is clear that different gravity field estimation approaches have different sensitivity to different error types.
In case of our method, the acceleration approach, the in-situ character of the observation equation ensures the effects of temporal aliasing are mitigated, while the integrating nature of the traditional variational equations approach accumulates \ac{AOD} model errors, magnifying their influence.
Irrespective of this difference sensitivity, all methods must rely on accurate orbits to properly build the normal matrices and estimate the gravity field coefficients.
We demonstrate that there are numerous error components resulting from orbit errors.
This work was the foundation for my thesis dissertation, where the effect of different components of orbit errors are quantified to an unprecedented level of accuracy, for the \ac{GRACE} mission.

My contribution to \cite{Gunter2011} for the first time explores the possibility of using numerous satellites equipped with \ac{GPS} receiver to augment dedicated gravimetric satellite missions.
We demonstrate that the current accuracy of \ac{GPS} observations is insufficient to provide meaningful improvements, unless a large number of satellites is used (in the order of hundreds).
However, future and multiple \ac{GNSS} systems should be accurate enough to make it possible to provide enough information to resolve sub-weekly to daily low-degree solutions with sufficient accuracy to observe, for the first time, global atmospheric mass transport processes.
This work has motivated my research into the design of a gravimetric CubeSats, specifically with efforts of pursing several funding applications to develop \ac{MEMS}-based space accelerometry.
% This would essentially mean that \ac{AOD} models could be replaced with these observations, with the associated benefits of a much more accurate and realistic description of high-frequency mass transport processes, and consequential improvements in the quality of gravimetric monitoring of lower-frequency processes, such as hydrology.


\cvheading{Highlighted publications}
\vspace{2ex}
\begin{refsection}
\nocite{
TeixeiraEncarnacao2016,
Ditmar2012,
Gunter2011,
}
\togglefalse{bbx:url}
\printbibliography
\end{refsection}

\end{cvtext}