% !TEX root = ./jgte_publication_summary.tex

\begin{cvtext}{Summary of selected publications}


In \cite{TeixeiradaEncarnacao2020} I report the results of massive numerical experiments testing different parametrization schemes.
A full-matrix scale factor parametrization of accelerometers onboard the \ac{GRACE} satellites  is able to handle the different data quality over the complete mission lifetime, while maintain consistency in the processing choices.
With this accelerometer parametrization scheme, it is possible reduce the number of bias parameters (removing the quadratic terms) and increase the scale parameters in the accelerometer calibration parametrization (from diagonal to full matrix and from monthly to daily), leading to a net decrease of 4 parameters per day, since that the y-axis bias parameters are estimated every 3 hours.
In terms of the quality of the gravity field model, there is a significant decrease in the intensity of the stripping artefacts, except for those periods with excellent data quality, where there is no noticeable decrease in the model’s quality relative to previous releases.
% To better support this optimization problem, I took advantage of meta-heuristic procedures, such as genetic algorithms, to speed up the convergence to an advantageous parametrization scheme.
From this research, both \ac{CSR} and \ac{JPL} (two of the main gravity field model producing centres) have modified their data processing to follow my recommendations.

In \cite{TeixeiraEncarnacao2019}, I report on the operational production of the Swarm gravity field models.
This research was done in cooperation with five international research institutes (
\acf{IfG} of the \acf{TUG},
\ac{ASU}  of the \acf{AVCR},
\acf{AIUB},
\dynhref{www.lr.tudelft.nl}{Aerospace Faculty} of the \ac{TUD} and
\acf{SES} of the \acf{OSU}),
within the context of a project initiated and lead by me.
I have taken the leading role securing funding from the \ac{ESA}, through the \ac{DISC}, to develop the methodology to best combine the individual gravity field solutions from each research institute. This project was awarded additional funding to routinely produce monthly models for the foreseeable future.
This publication capitalizes on the experience gathered from producing 6 years of Swarm gravity field models, illustrating the strengths and limitations of this dataset.
I demonstrate that the published models are of superior quality than any of the individual models that are used to estimate it, and that a smoothing of 750km is needed over land, while 3000km is needed over the oceans.

In \cite{TeixeiraEncarnacao2016}, the habitability of the Swarm satellites to observe changes on Earth's gravity field was for the first time demonstrated.
The Swarm gravity field models constitute an independent source of gravimetric data from dedicated mission (\ac{GRACE} and \ac{GRACE-FO}).
Although their spatial resolution is much lower, the Swarm models are still able to describe monthly variations of Earth's gravity field within a few millimetres geoid height, when compared with \ac{GRACE}.
This has proven to be particularly beneficial in monitoring mass transport processes during the gap between the periods when \ac{GRACE} and \ac{GRACE-FO} were and are (respectively) collecting data.

My contribution to \cite{Ditmar2012} demonstrates that the influence of temporal aliasing caused by errors in the \ac{AOD} models, needed to remove rapid-changing gravity variations that are impossible for \ac{GRACE} to observe, cannot explain the unexpected low quality of the gravity field models, compared with pre-launch predictions, particularly in what concerns the striping artefacts.
Additionally, the results of this publication contrast with other similar studies in the sense that it is clear that different gravity field estimation approaches have different sensitivity to different error types.
In case of our method, the acceleration approach, the in-situ character of the observation equation ensures the effects of temporal aliasing are mitigated, while the integrating nature of the traditional variational equations approach accumulates \ac{AOD} model errors, magnifying their influence.
Irrespective of this difference sensitivity, all methods must rely on accurate orbits to properly build the normal matrices and estimate the gravity field coefficients.
We demonstrate that there are numerous error components resulting from orbit errors.
This work was the foundation for my thesis dissertation, where the effect of different components of orbit errors are quantified to an unprecedented level of accuracy, for the \ac{GRACE} mission.

My contribution to \cite{Gunter2011} for the first time explores the possibility of using numerous satellites equipped with \ac{GPS} receiver to augment dedicated gravimetric satellite missions.
We demonstrate that the current accuracy of \ac{GPS} observations is insufficient to provide meaningful improvements, unless a large number of satellites is used (in the order of hundreds).
However, future and multiple \ac{GNSS} systems should be accurate enough to make it possible to provide enough information to resolve sub-weekly to daily low-degree solutions with sufficient accuracy to observe, for the first time, global atmospheric mass transport processes.
This work has motivated my research into the design of a gravimetric CubeSats, specifically with efforts of pursing several funding applications to develop \ac{MEMS}-based space accelerometry.
% This would essentially mean that \ac{AOD} models could be replaced with these observations, with the associated benefits of a much more accurate and realistic description of high-frequency mass transport processes, and consequential improvements in the quality of gravimetric monitoring of lower-frequency processes, such as hydrology.


\cvheading{Highlighted publications}
\vspace{2ex}
\begin{refsection}
\nocite{
TeixeiradaEncarnacao2020,
TeixeiraEncarnacao2019,
TeixeiraEncarnacao2016,
Ditmar2012,
Gunter2011,
}
\togglefalse{bbx:url}
\printbibliography
\end{refsection}

\end{cvtext}