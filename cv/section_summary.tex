% !TEX root = ./jgte_summary.tex

\begin{cvtext}{Summary}

% https://getpocket.com/explore/item/27-words-you-should-never-use-to-describe-yourself
% 27 Words You Should Never Use to Describe Yourself
% "Innovative"
% "World-class"
% "Driven"
% "Extensive experience"
% "Authority"
% "Results-oriented"
% "Responsible"
% "Global provider"
% "Motivated"
% "Creative"
% "Track record"
% "Organizational"
% "Dynamic"
% "Guru"
% "Curator"
% "Passionate"
% "Unique"
% "Incredibly..."
% "Serial Entrepreneur"
% "Strategic"
% "Collaborative"


% 150 words summary:
% I study ways of exploiting the maximum resolution and accuracy of the measurements collected by the GRACE satellites. My work focuses on a) the calibration of the accelerometers, particular relevant after 2011, when the thermal control on the satellites was switched off, b) testing large number of unconventional parametrization schemes, c) developing time-series analysis methods and processing suitable gravimetric data to predict the long-term trends in the GRACE gravity field models over the GRACE/GRACE-FO gap, and d) developing novel methods of connecting the L1B data directly to parameters describing hydrological, solid Earth and glaciological models.
% In additional to my geophysical interests, I am an aerospace engineer. I worked in numerous areas, including Structural Mechanics, Aerodynamics, Preliminary Vehicle Design, Single Stage to Orbit and Laser Propulsion, which have given me the opportunity to broaden my understanding of Physics and Satellite System Engineering.

% Situation: In my work (general)
% Task: I have done something (specific, what is expected)
% Action: i did it with this approach
% Results: and this is how i succeeded

In my research, I look for ways to measure Earth's gravity changes with satellites.
With a background in Aerospace Engineering, I focus on Space Accelerometry, Estimation Theory and satellite data processing.
My main research objective is to devise data processing and modelling strategies that maximize the temporal resolution of the gravimetric data collected by LEO satellites, as well as exploiting small satellites for gravimetric and thermospherometric purposes.

% current situation
As assistant professor at the Delft University of Technology, I study ways of exploiting the maximum resolution and accuracy of the measurements collected by the \ac{GRACE} satellites, as well as exploiting small satellites for gravimetric and thermospherometric purposes.

%project: G-Swarm
I manage a \dynhref{https://www.researchgate.net/project/Multi-approach-gravity-field-models-from-Swarm-GPS-data}{research project} that exploits the \ac{GPS} data from the Swarm satellites to describe the temporal variations of Earth's gravity, describing the hydrological cycle and climatological trends over river basins and Polar Regions.
I took the lead in coordinating with several European and US institutes (%
Institute of Geodesy of the Graz University of Technology -- Austria,
Astronomical Institute of the Academy of Sciences of the Czech Republic,
Astronomical Institute of the University of Bern -- Switzerland,
Faculty of Aerospace Engineering of the Delft University of Technology -- the Netherlands -- and
School of Earth Sciences of the Ohio State University -- USA%
) the cooperation initiative for the research and promotion of Swarm's gravity field models, which eventually lead to being funded by the Swarm \ac{DISC} consortium and will continue in the coming years.
This project has moved from a development phase, where we tested the added value of inter-satellite baselines estimated from GPS data and different options for the modelling or observation of the non-gravitational accelerations, into an operational stage, where the Swarm gravity fields will be distributed quarterly to the scientific community.

%project: uPGRADE
In cooperation with \ac{CSR}, I manage the uPGRADE project, where we aim at measuring the movement of water in the Earth’s near surface, at regional scale, by sensing the minute changes in our planet’s gravity from its 500km orbit.
The uPGRADE satellite will contribute to the monitoring of surface mass transport processes along with other gravimetric satellite missions such as GRACE-FO and Swarm.
It will also be able to measure the neutral density and cross-track winds in the thermosphere, contributing to the study of the effect of solar activity on that environment and perfecting the drag models necessary to accurately predict the consequences of the sharp increase of space debris.
Although most systems in uPGRADE are Commercial Off-The-Shelf (COTS), such as the dual-bad GNSS-receiver, communications, magneto-torquers and propulsion, we are developing the star-tracker and the space accelerometer based on Micro Electro-Mechanical System (MEMS).
The latter is being designed specifically for the gravimetric application; by taking advantage of the micro-scale physics in MEMS devices and carefully designed low-noise electronics, it may be able to measure non-gravitational accelerations down to a few mHz, far below what traditional electrostatic accelerometers are able to measure.

%at CSR
As a Scientist Associate at \ac{CSR} of the \ac{UTexas}, my work focusesed on:
\begin{itemize}[topsep=0pt,itemsep=1pt,parsep=0pt,partopsep=0pt]
\item the calibration of the accelerometers, particular relevant after 2011, when the thermal control on the satellites was switched off;
\item testing large number of unconventional parametrization schemes;
% \item developing time-series analysis methods and processing suitable gravimetric data to predict the long-term trends in the \ac{GRACE} gravity field models over the \ac{GRACE}\slash\ac{GRACE-FO} gap; and
\item processing the \ac{GOCE} data in preparation for the GGM07 static gravity field mode.
\end{itemize}

%before CSR
As a Post-doctoral Fellow at \ac{TUD}, I dedicated my efforts to implement the Level 2 data processing facility of the \dynhref{https://earth.esa.int/web/guest/missions/esa-operational-eo-missions/swarm}{Swarm satellite mission}, concerning the Precise Orbit Determination and Thermospheric Neutral Density processing streams.
I have acquired expertise in \ac{DSP} techniques and contributed to the processing of Swarm accelerometer data, by combining non-gravitational accelerations derived from \ac{GPS} data and the accelerometer measurements.
In doing so, I have greatly removed the long-term bias in the accelerometer data.
During this time, I also matured my skills in data management and automated processing.

%PhD expertise
During my PhD, I have worked with different types of satellite gravimetric data, namely \acl{hlsst} (\aclp{KO}), \acl{llsst} (\aclp{ISR}), and \acl{SGG} (differential accelerometer measurements).
My PhD research focused on:
\begin{itemize}[topsep=0pt,itemsep=1pt,parsep=0pt,partopsep=0pt]
\item modelling the data errors accurately and how its amplitude and spectra influences the quality of the resulting gravity field models;
\item quantifying the error budget of future gravimetric satellite missions, to an unprecedented level of detail;
\item analysing several mission concepts and modelled their error budget in terms of the observations and gravity field parameters;
% \item establishing that a constellation of numerous non-dedicated satellites make it possible to measure fast mass transport processes;
\item demonstrating that some mission concepts (those with large radial distances, \acs{e.g.} the cartwheel formation) are very sensitive to particular types of errors (specifically errors connected with \ac{GPS} observations); %
\item proved that alternative mission concepts (the cross-track pendulum formation) are much better suited to complement planned future gravimetric missions.%
\end{itemize}
This has allowed me to study future gravimetric missions in detail, even unconventional ones such as augmenting dedicated gravimetric missions with a large constellation of non-dedicated satellites.
My expertise on this topic has been noted by peers, who have invited me to participate in numerous research projects involving international teams.


% %project: GRACE CubeSat
% High in my research priorities is to find ways of increasing the temporal resolution of gravimetric data, while reducing the cost of maintain the continued monitoring of the Earth System.
% Data collected by multiple satellites systems would address the problem of temporal aliasing, which cannot be mitigated with a single observing system.
% Furthermore, the geophysical processes taking place from sub-orbital to sub-weekly periods can be better understood if gravimetric data is able to describe them accurately.
% These objectives are best achieved with the augmentation of dedicated missions.
% My research looks into the development of a CubeSat architecture that replicates the gravimetric capabilities of the \ac{GRACE} satellites while identifying key technology gaps.
% These activities are the initial steps towards the objective of developing a complete suite of capable miniaturized sensors, in order to mature the technology required for a \enquote{CubeSat-GRACE}.

% %project: Direct gravimetric data assimilation
% My research also aims at exploiting satellite gravimetric data to better constrain regional geophysical models, in particular those related to hydrology and glaciological studies.
% In this data assimilation approach, the gravimetric data collected by the \ac{GRACE} and \ac{GRACE-FO} satellites are linked to parameters that drive spatial and time-variable aspects in the geophysical models, instead of the traditional process of considering monthly gravity field models.
% This allows for the particular dynamics of a geographical region to be parametrised in the most realistic way, when building the model describing the dominant geophysical process that takes place over that region.
% For example, a particular hydrological storage catchment might be loaded with rainwater homogeneously over the wet season and gradually drain in such a way that water mass collects towards the outlet along the associated river system.
% The same idea can be applied to the drainage basin of large glaciers, although the intrinsic temporal scale is much larger.
% The intrinsically inhomogeneous distribution and evolution of mass in space and time, which is known from (\ac{e.g.}) ground measurements, can be constrained with \ac{GRACE} data because these data observe and represent such evolutions (if large enough).
% In practice, the spatial and temporal variations are described by model parameters that represent mass changes with assumed and predefined spatial and temporal functions, which are fitted to the gravimetric data.
% Extreme events such as floods are properly represented by sharp functions super-imposed to the undisturbed seasonal variations, with coefficients that unequivocally describe onset speed, mass flow (and consequently average water height) and drainage delays, to which (\ac{e.g.}) risk of infrastructure damage can be directly correlated.
% This approach circumvents the weaknesses of the traditional monthly models, most notably spatial leakage and aliasing of other high-frequency geophysical signals, and takes full advantage of the data's spatial and temporal information to improve our understanding of large geophysical processes.

%background expertise
I have studied and worked in numerous areas, including Structural Mechanics, Aerodynamics, Preliminary Vehicle Design, Single Stage to Orbit and Laser Propulsion, which have given me the opportunity to broaden my understanding of Physics.
I am an avid programmer, actively learning new languages and techniques in order to better implement the algorithms and procedures required to develop my research.
I openly share the code I develop in \dynhref{http://github.com/jgte}{GitHub}.

\end{cvtext}