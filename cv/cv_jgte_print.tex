\documentclass[a4paper,12pt]{article}

% !TEX root = ./jgte_cv.tex

% sudo tlmgr install carlito fontaxes footmisc tabu varwidth needspace

% http://www.tug.dk/FontCatalogue/carlito/
\usepackage[sfdefault,lf]{carlito}
\renewcommand*\oldstylenums[1]{\carlitoOsF #1}
%% The 'lf' option for lining figures
%% The 'sfdefault' option to make the base font sans serif

\usepackage[T1]{fontenc}
\usepackage[utf8]{inputenc}    % utf8 support       %!!!!!!!!!!!!!!!!!!!!
\DeclareUnicodeCharacter{301}{'}
\usepackage{lmodern}
\usepackage{eurosym}
% \usepackage{lastpage}
% \usepackage{xspace}
\usepackage[a4paper,left=1in,right=1in,top=1in,bottom=1in,headsep=1em,footskip=1em]{geometry}
% \addtolength{\oddsidemargin}{-1.5cm}
% \addtolength{\evensidemargin}{-1.5cm}
% \addtolength{\textwidth}{3cm}
% \addtolength{\topmargin}{-1.5cm}
% \addtolength{\textheight}{2.5cm}
% \usepackage{graphicx}
\usepackage{multirow}
\usepackage{array}
% \usepackage[usenames,dvipsnames,table]{xcolor}
\usepackage{csquotes}
% \usepackage{pgfgantt}
% \usepackage{etoolbox}
\usepackage{hyperref}
\usepackage{amsfonts,amsmath,amssymb}
% \usepackage{soul} % for smarter (word-wrapping) underlining.
% \usepackage{lineno}
\usepackage{enumitem} % for broken enumerate in ethics section.
% \usepackage{multirow}
% \usepackage{float}
\usepackage[document]{ragged2e}
\usepackage{longtable}[=v4.13]
\usepackage{tabu}
\usepackage{needspace}

% ----------------------------------

\usepackage{url}                            % http://www.ctan.org/pkg/url
\makeatletter
\g@addto@macro{\UrlBreaks}{\UrlOrds}
\g@addto@macro{\UrlBreaks}{%
  \do\/\do\a\do\b\do\c\do\d\do\e\do\f%
  \do\g\do\h\do\i\do\j\do\k\do\l\do\m%
  \do\n\do\o\do\p\do\q\do\r\do\s\do\t%
  \do\u\do\v\do\w\do\x\do\y\do\z%
  \do\A\do\B\do\C\do\D\do\E\do\F\do\G%
  \do\H\do\I\do\J\do\K\do\L\do\M\do\N%
  \do\O\do\P\do\Q\do\R\do\S\do\T\do\U%
  \do\V\do\W\do\X\do\Y\do\Z}
\makeatother
\Urlmuskip=0mu plus 1mu\relax
% !TEX root = ./cv_jgte.tex

% \usepackage[sorting=ydnt,backend=biber,style=authoryear,maxnames=99]{biblatex}
% \addbibresource{cv_jgte.bib}

\usepackage[style=verbose,backend=bibtex,isbn=false,sorting=none,maxbibnames=99,sorting=ydnt,style=authoryear]{biblatex}
\addbibresource{library.bib}
\addbibresource{cv_jgte.bib}
% \AtEveryCitekey{\clearfield{title}}
\AtEveryCitekey{%
  \ifentrytype{article}{%
  }{%
    \clearfield{url}%
    \clearfield{urldate}%
  }%
}
\AtEveryCitekey{%
  \ifentrytype{inproceedings}{%
  }{%
    \clearfield{url}%
    \clearfield{urldate}%
  }%
}

% ----------------------------------

\usepackage{xstring}
\newboolean{bold}
\newcommand{\makeauthorsbold}[1]{%
  \DeclareNameFormat{author}{%
  \setboolean{bold}{false}%
    \renewcommand{\do}[1]{\expandafter\ifstrequal\expandafter{\namepartfamily}{####1}{\setboolean{bold}{true}}{}}%
    \docsvlist{#1}%
    \ifthenelse{\value{listcount}=1}
    {%
      {\expandafter\ifthenelse{\boolean{bold}}{\mkbibbold{\namepartfamily\addcomma\addspace \namepartgiveni}}{\namepartfamily\addcomma\addspace \namepartgiveni}}%
    }{\ifnumless{\value{listcount}}{\value{liststop}}
      {\expandafter\ifthenelse{\boolean{bold}}{\mkbibbold{\addcomma\addspace \namepartfamily\addcomma\addspace \namepartgiveni}}{\addcomma\addspace \namepartfamily\addcomma\addspace \namepartgiveni}}%
      {\expandafter\ifthenelse{\boolean{bold}}{\mkbibbold{\addcomma\addspace \namepartfamily\addcomma\addspace \namepartgiveni\addcomma\isdot}}{\addcomma\addspace \namepartfamily\addcomma\addspace \namepartgiveni\addcomma\isdot}}%
      }
    \ifthenelse{\value{listcount}<\value{liststop}}
    {\addcomma\space}{}
  }
}

\makeauthorsbold{%
Encarna{\c{c}}{\~{a}}o,Jo{\~{a}}o,%
Jo{\~{a}}o Encarna{\c{c}}{\~{a}}o,
de Teixeira da Encarna{\c{c}}{\~{a}}o,%
Teixeira Encarna{\c{c}}{\~{a}}o,%
Teixeira da Encarna{\c{c}}{\~{a}}o,%
da Encarna{\c{c}}{\~{a}}o,Jo{\~{a}}o Teixeira,%
Teixeira da Encarnacao,%
Encarnacao,%
Joao%
}
% !TEX root = ./cv_jgte.tex

% Set your name here
\def\name{Jo\~ao~de~Teixeira~da~Encarna\c c\~ao}


% Replace this with a link to your CV if you like, or set it empty
% (as in \def\footerlink{}) to remove the link in the footer:
\def\footerlink{}

% The following metadata will show up in the PDF properties
\hypersetup{
  colorlinks = true,
  urlcolor = blue,
  pdfauthor = {\name},
  pdfkeywords = {},
  pdftitle = {\name: Curriculum Vitae},
  pdfsubject = {Curriculum Vitae},
  pdfpagemode = UseNone
}

\author{\name}
% !TEX root = ./jgte_cv.tex

% ----------------------------------

\usepackage{titlesec}% Used to customize the \section command
\titleformat{\section}
[hang]      % shape
{\sf\bf\normalsize} % format
{}          % label
{0pt}       % sep
{\uppercase}          % before-code
\titlespacing{\section}{0pt}{1ex plus 0.5ex minus 0.5ex}{1ex plus 0.5ex minus 0.5ex} % Spacing around sections
\titleformat{\subsection}
[hang]      % shape
{\sf\bf\normalsize} % format
{}          % label
{0pt}       % sep
{}          % before-code
\titlespacing{\subsection}{0pt}{1ex plus 0.5ex minus 0.5ex}{1ex plus 0.5ex minus 0.5ex} % Spacing around sections

\renewcommand{\bottomtitlespace}{1cm}

% ----------------------------------

\usepackage{fancyhdr}
\pagestyle{fancy}
\fancyhf{}
\rhead{\it{\name}}
\lhead{}
\rfoot{\thepage~of \pageref*{endpage}}
\thispagestyle{empty}

% ----------------------------------

\newcommand{\cvheading}[1]{
  \section*{#1}
  % \setlength{\parindent}{3ex}
}

%this version of the command above adds a space after the heading, can be used by itself (see publications)
\newcommand{\cvsubheading}[1]{
  \subsection*{#1}
  % \setlength{\parindent}{3ex}
}

\newenvironment{cvsection}[1]{
  % \setlength{\floatsep}{0pt}
  % \setlength{\textfloatsep}{0pt}
  % \setlength{\intextsep}{0pt}
  \cvheading{#1}
  \renewcommand{\arraystretch}{1.5}
  % https://tex.stackexchange.com/questions/110266/set-longtable-to-fit-the-page-width
  \begin{longtabu} to \textwidth {lX}
}{
  \end{longtabu}
  % \vspace{-1.5ex}
  \raggedbottom
}

\newenvironment{cvsectionlist}[1]{
  % \setlength{\floatsep}{0pt}
  % \setlength{\textfloatsep}{0pt}
  % \setlength{\intextsep}{0pt}
  \cvheading{#1}
  \renewcommand{\arraystretch}{0}
  % https://tex.stackexchange.com/questions/110266/set-longtable-to-fit-the-page-width
  \begin{longtabu} to \textwidth {lX}
}{
  \end{longtabu}
  % \vspace{-1.5ex}
  \raggedbottom
}

\newenvironment{cvtext}[1]{
  % \vspace{0.15in}
  \cvheading{#1}
  \vspace{1em}
  \begin{justify}
  % \vspace{1.5ex}
  % \begin{minipage}{\textwidth}

  % \raggedright
}{
  \end{justify}
  % \end{tabular*}
  % \end{tabularx}
  % \end{minipage}
}

\flushbottom

% ----------------------------------

\newcommand\cvsectionline[3]{
  \begin{tabular}{l}
  #1 -- #2 \\
  \end{tabular}
  &
  #3\\
}

\newcommand\cvsectionlineone[2]{
  \begin{tabular}{l}
  #1 \\
  \end{tabular}
  &
  #2\\
}

% ------ tight lists --------------------------------------------------------

\newlist{itemizetight}{itemize}{2}
\setlist[itemizetight]{nosep,leftmargin=0pt,label=}

\newenvironment{itti}{
  \vspace*{-1.5ex}
  \begin{itemizetight}
}{
  \end{itemizetight}
}

\newenvironment{ittj}{
  \begin{itemizetight}
}{
  \end{itemizetight}
  \vspace*{1ex}
}


\newenvironment{ittib}{
  \begin{itemizetight}[label=$\bullet$,leftmargin=4ex]
}{
  \end{itemizetight}
}


% ------ dynamic web references -------------------------------

%this needs to come before the acronyms
\newtoggle{expliciturl}
\toggletrue{expliciturl}
\newcommand{\dynhref}[2]{%
  \iftoggle{expliciturl}{%
    #2 \footnote{\href{http://#1}{\detokenize{#1}}}%
  }{%
    \href{http://#1}{#2}%
  }%
}

%acronym packaged loaded inside
% !TEX root = ./jgte_cv.tex

%This is specficit to the CV
\iftoggle{expliciturl}{%
  \usepackage[nohyperlinks,nolist,footnote]{acronym}
}{%
  \usepackage[nohyperlinks,nolist]{acronym}
}%

\newcommand*{\urlhttp}[1]{\href{http://#1}{\detokenize{#1}}}

\ifdefined\UseSimpleAcronyms
  \newcommand{\acrodefcite}[4][]{\acrodef{#2}{#3}}
  \newcommand{\acrocite}[4][]{\acro{#2}{#3}}
  \newcommand{\acroref}[3]{\acro{#1}{#2}}
  \newcommand{\acrourl}[3]{\acro{#1}{#2}}
  \newcommand{\acrourlc}[4]{\acro{#2}{#3}}

\else
	%Helper for the definition of acronyms with citations:
	%\acrocite{acronym}{long version}{citation key}
	%The acronyms defines a citation in the list of acronyms.
	%A second acronym called acronym.cite includes the citation in the text.
	%example:
	%\acrocite{FES2004}{Finite Element Solution model release 2004}{Lyard2006}
	%produces:
	%\acro{FES2004}{Finite Element Solution model release 2004\acroextra{, \citep{Lyard2006}}}
	%\acrodef{FES2004.cite}[!use acl with .cite acronyms!]{Finite Element Solution model release 2004 (FES2004, \citealt{Lyard2006})}
	%WARNING: always use \acl with the .cite acronyms!

	\ifdefined\UseCitationsInAcronyms
    \newcommand{\acrodefcite}[4][]{%
      \acrodef{#2.cite}[!use acl with .cite acronyms!]{#3 (#2)#1 \protecting{(\citealt{#4})\acused{#2}}}%
      \acrodef{#2.citeonly}[!use acl with .citeonly acronyms!]{\protecting{(\citealt{#4})}}%
		}
    \newcommand{\acrocite}[4][]{%
      \acro{#2}{#3\acroextra{#1, \protecting{\citet{#4}}}}%
      \acrodefcite[#1]{#2}{#3}{#4}%
		}
	\else
    \newcommand{\acrodefcite}[4][]{%
      \acrodef{#2.cite}[!use acl with .cite acronyms!]{#3 (#2)#1\protecting{\acused{#2}}}%
		}
    \newcommand{\acrocite}[4][]{%
      \acro{#2}{#3}%
      \acrodefcite[#1]{#2}{#3}{#4}%
		}
	\fi

	%Helper for the definition of acronyms with references:
	%\acrocite{acronym}{long version}{reference key}
	%The acronyms defines a reference in the list of acronyms.
	%A second acronym called acronym.ref includes the reference in the text.
	%example:
	%\acrocite{HRF}{Hill Reference Frame}{sec:hrf}
	%produces:
	%\acro{HRF}{Hill Reference Frame\acroextra{, \ref{sec:hrf}}}
	%\acrodef{HRF.ref}[]{Hill Reference Frame (HRF, \ref{sec:hrf})}
	%WARNING: always use \acl with the .ref acronyms!
	\ifdefined\UseReferencesInAcronyms
    \newcommand{\acroref}[3]{%
      \acro{#1}{#2\acroextra{, \begin{sloppypar}\protecting{\ref{#3}\end{sloppypar}}}}%
      \acrodef{#1.ref}[!use acl with .ref acronyms!]{#2 \protecting{(#1, \ref{#3})\acused{#1}}}%
		}
	\else
    \newcommand{\acroref}[3]{%
      \acro{#1}{#2}%
      \acrodef{#1.ref}[!use acl with .ref acronyms!]{#2 (#1)\protecting{\acused{#1}}}%
		}
	\fi

  %specific to this document
  \let\UseUrlInAcronyms

  \ifdefined\UseUrlInAcronyms
    %includes the URL link in the short form of the acronym
    \newcommand{\acrourl}[3]{%
      \iftoggle{expliciturl}{%
        \acro{#1}[\href{http://#3}{#1}]{#2, \protecting{\urlhttp{#3}}}%
      }{%
        \acro{#1}[\href{http://#3}{#1}]{#2\acroextra{, \protecting{\urlhttp{#3}}}}%
      }%
      \acrodef{#1.url}[!use acl with .url acronyms!]{#2 \protecting{(#1, \urlhttp{#3})\acused{#1}}}%
      \acrodef{#1.urlonly}[!use acl with .urlonly acronyms!]{\protecting{\urlhttp{#3}}}%
}
    \newcommand{\acrourlc}[4]{%
      \iftoggle{expliciturl}{%
        \acro{#2}[\href{http://#4}{#1}]{#3, \protecting{\urlhttp{#4}}}%
      }{%
        \acro{#2}[\href{http://#4}{#1}]{#3\acroextra{, \protecting{\urlhttp{#4}}}}%
      }%
      \acrodef{#2.url}[!use acl with .url acronyms!]{#3 \protecting{(#2, \urlhttp{#4})\acused{#2}}}%
      \acrodef{#2.urlonly}[!use acl with .urlonly acronyms!]{\protecting{\urlhttp{#4}}}%
}
  \else
    %the link text is only available in the list of acronyms
    \newcommand{\acrourl}[3]{%
      \acro{#1}{#2\acroextra{, \protecting{\urlhttp{#3}}}}%
      \acrodef{#1.url}[!use acl with .url acronyms!]{#2 \protecting{(#1, \urlhttp{#3})\acused{#1}}}%
      \acrodef{#1.urlonly}[!use acl with .urlonly acronyms!]{\protecting{\urlhttp{#3}}}%
    }
    \newcommand{\acrourlc}[4]{%
      \acro{#2}[#1]{#3\acroextra{, \protecting{\urlhttp{#4}}}}%
      \acrodef{#2.url}[!use acl with .url acronyms!]{#3 \protecting{(#2, \urlhttp{#4})\acused{#2}}}%
      \acrodef{#2.urlonly}[!use acl with .urlonly acronyms!]{\protecting{\urlhttp{#4}}}%
    }
  \fi
\fi
\acresetall

\newcommand{\unit}[2]{\ensuremath{{\sf#1}\thickspace{\sf#2}}}

\begin{acronym}[------------------------]

\renewcommand{\baselinestretch}{0.8}%
\small

%don't use \ace, \act or \acf with acronyms that include a \citep command
%linguistic short forms
\acrodef{a.i.}{\emph{ad interim}\acroextra{, temporarily}}
\acro{a.o.}{amongst others}
\acrodef{ca.}{\emph{circa}\acroextra{, approximately}}
\acrodef{cf.}{\emph{confer}\acroextra{, compare}}
\acrodef{e.g.}{\emph{exempli gratia}\acroextra{, for example}}
\acrodef{etc.}{\emph{et cetera}\acroextra{, and so forth}}
\acro{i.a.}{\emph{inter alia}\acroextra{, amongst others}}
\acrodef{i.e.}{\emph{id est}\acroextra{, that is}}
\acro{i.o.}{\emph{in illo ordine}\acroextra{, respectively}}
\acrodef{p.}{page}
\acrodef{pp.}{pages}
\acrodef{vs.}{\emph{versus}\acroextra{, against}}

%Titles
\acrodef{Prof.}{Professor}
\acrodef{Dr.}{Doctor}

%#
\acro{1D}{uni-dimensional}
\acro{2D}{bi-dimensional}
\acro{3D}{three-dimensional}
%A
\acrocite{AA}{Acceleration Approach}{Rummel1979}
\acrocite{DAA}{Decorrelated \acl{AA}}{Bezdek2014}
\acro{ACC}{Accelerometer}
\acro{ASCR}{Academy of Sciences of the Czech Republic}
\acro{ACS}{Attitude Control System}
\acro{ADP}{Auxiliary Data Provider}
\acro{ADC}{Analogue-to-Digital Converter}
\acrourl{AGU}{American Geophysical Union}{sites.agu.org}
\acro{AIL}{Action Item List}
\acrourl{AIUB}{Astronomical Institute of the University of Bern\acroextra{, Switzerland}}{www.aiub.unibe.ch}
\acrocite{AIUB-GRACE02S}{\acs{AIUB} \acs{GRACE}-only model, version 2}{Jaggi2012}
\acrocite{AIUB-GRACE03S}{\acs{AIUB} \acs{GRACE}-only model, version 3}{Jaggi2012}
\acro{aka}{also known as}
\acro{AOCS}{Attitude and Orbital Control System}
\acro{AOD}{Atmosphere and Ocean De-aliasing}
%TODO : go throught the text and make sure any \ac{AOD1B} instances is followed by 'product'.
\acrocite[ product]{AOD1B}{Atmosphere and Ocean De-aliasing \acl{L1B}}{Flechtner2006,Flechtner2007,Flechtner2011}
\acrocite{AOTIM-5}{Arctic Ocean Tidal Inverse Model}{Padman2010}
\acro{ANGARA}{Analysis of Non-Gravitational Accelerations due to Radiation pressure and Aerodynamics}
\acro{AO}{Announcement of Opportunity}
\acro{APDF}{Archiving and Payload Data Facility}
\acro{ARMA}{Auto-Regressive Moving-Average}
% \acro{AS}{Anti-Spoofing}
\acrourl{ASAP}{Austrian Space Applications Program}{www.ffg.at/en/austrian-space-applications-programme}
\acro{ASCII}{American Standard Code for Information Interchange}
\acro{ASD}{Amplitude Spectral Density\acroextra{, equal to the square-root of the \acs{PSD}}}
\acrourl{AS}{Astrodynamics and Space missions\acroextra{, Faculty of Aerospace Engineering, \acs{TUD}}}{www.as.lr.tudelft.nl/}
\acrourl{ASU}{Astronomical Institute\acroextra{ (Astronomick\'y \'ustav), \acs{AVCR}}}{www.asu.cas.cz/en}
\acrourl{AVCR}{Czech Academy of Sciences\acroextra{ (Akademie v\v{e}d \v{C}esk\'e Republiky), Czech Republic}}{www.avcr.cz/en/}
%B
\acrourl{BASH}{Bourne-again shell}{www.gnu.org/software/bash}
\acrocite{BDNSS}{BeiDou\slash Compass Navigation Satellite System}{Chengzhi2013}
\acrocite{Bernese}{Bernese \acs{GNSS} software}{Dach2015}
\acrourl{BGC}{Bristol Glaciology Centre}{www.bristol.ac.uk/geography/research/bgc/}
\acrourl{BIT}{Beijing Institute of Technology}{english.bit.edu.cn}
\acro{BIRP}{Background Intellectual Property Rights}
%C
\acro{C/A-code}{Coarse\slash Acquisition code}
\acro{CAD}{Computer-Aided Design}
\acrocite{CADA00.10}{Circum-Antarctic Tidal Simulation Inverse Model}{Padman2010a}
\acro{Cal/Val}{Calibration and Validation}
\acro{CAKAO}{Combined analysis of kinematic orbits and loading observations to determine mass redistribution\acroextra{, \"Osterreichische Forschungsf\"orderungsgesellschaft mbH}}
\acrourl{CAS}{Chinese Academy of Sciences\acroextra{, China}}{english.cas.cn}
\acro{COAS}{College of Oceanic and Atmospheric Sciences\acroextra{, Oregon State University}}
\acro{CCDB}{Characterisation and Calibration Database}
\acro{CCR}{Corner-Cube Retro-reflector}
\acrourl{CDF}{Common Data Format}{cdf.gsfc.nasa.gov}
\acrourl{CERGA}{Centre d'{\'{e}}tudes et de Recherches en G{\'{e}}odynamique et Astrom{\'{e}}trie}{wwwrc.obs-azur.fr/cerga/CERGA_anglais.html}
\acrourl{CLS}{Collecte Localisation Satellites}{www.cls.fr}
\acrocite{CHAMP}{CHallenging Mini-Satellite Payload}{Reigber1996,Reigber2002}
\acro{CODE}{Centre for Orbit Determination in Europe}
\acro{CIF}{Conventional Inertial Frame}
\acrocite{CMA}{Celestial Mechanics Approach}{Beutler2010a}
\acrourl{CNES}{Centre National d{\'{E}}tudes Spatiales\acroextra{, France}}{cnes.fr}
\acro{CoM}{Centre of Mass}
\acrourl{CICERO}{Community Initiative for Continuing Earth Radio Occultation}{geooptics.com}
\acrocite{CNES/GRGS-10d}{\acs{CNES}\slash\acs{GRGS} 10-days gravity field models}{Lemoine2007,Bruinsma2010,Lemoine2013}
\acrourl{CNRS}{Centre National de la Recherche Scientifique}{www.cnrs.fr/index.php}
\acro{COSMIC}{Constellation Observing System for Meteorology, Ionosphere and Climate\acroextra{ satellite mission (see  also \acs{F3C})}} %reference added to F3C.cite
\acro{COSMIC-2}{2nd Constellation Observing System for Meteorology, Ionosphere and Climate\acroextra{  satellite mission (see also \acs{F7C2})}} %reference added to F3C.cite
\acrourl{COSPAR}{Committee on Space Research}{www.cospar-assembly.org}
\acro{COTS}{Commercial Off-The-Shelf}
\acro{CPR}{Cycle Per Revolution}
\acrodefplural{CPR}{Cycles Per Revolution}
\acro{CPU}{Central Processing Unit}
\acrourl{CSR}{Center for Space Research\acroextra{, \acs{UTexas}, \acs{USA}}}{www.csr.utexas.edu}
\acrocite{CSR3.0}{version 3 of the \acs{CSR}'s global ocean model}{Eanes1996}
\acrocite{CSR4.0}{version 4 of the \acs{CSR}'s global ocean model}{Eanes1996}
\acroref{CRF}{Celestial Reference Frame}{sec:crf} %if you change this, also change the superscript
\acro{CTF}{Conventional Terrestrial Frame}
\acrourl{CTU}{Czech Technical University in Prague}{www.cvut.cz/en}
\acrodef{CV}{Curriculum Vit\ae}
\acrodefplural{CV}{Curricula Vit\ae}
%D
\acrocite{DAA}{Decorrelated \acl{AA}}{Bezdek2014,Bezdek2016}
\acro{DAS}{Degree Amplitude Spectrum}
\acrodefplural{DAS}{Degree Amplitude Spectra}
\acro{DD}{Double-differenced}
\acrocite{DEMETER}{Development of a European Multi-model Ensemble system for seasonal to inTERannual prediction}{Palmer2004}
\acro{DEOS}{Delft Institute for Earth-Oriented Space research}
\acro{DFACS}{Drag-Free Attitude Control Systems}
\acrourl{DLR}{Deutsches Zentrum f\"ur Luft- und Raumfahrt\acroextra{, Germany}}{www.dlr.de}
\acro{DGFI}{Deutsches Geod\"atisches Forschungsinstitut}
\acrocite{DGM-1S}{Delft Gravity Model}{Farahani2013}
\acro{DISC}{Data, Innovation and Science Cluster}
%TODO : go throught the text and make sure any \ac{DMT} instances is followed by 'product' or 'model'.
\acro{DMP}{Data Management Plan}
\acrocite[ model]{DMT}{Delft Mass Transport}{Liu2010,Ditmar2013}
\acro{DoF}{Direction of Flight}
\acrourl{DOI}{Digital Object Identifier}{www.doi.org}
\acrocite{DORIS}{Doppler Orbit Determination and Radio-positioning Integrated on Satellite}{Dorrer1991,Barlier2005,Willis2006}
\acro{DOWR}{Dual One-Way Ranging}
\acro{DoY}{Day of Year}
\acro{DQW}{Data Quality Workshop}
\acro{DSP}{Digital Signal Processing}
\acrocite{DTM}{Drag Temperature Model}{Bruinsma2003}
\acrourl{DTU}{Danish Technical University\acroextra{, Denmark}}{www.dtu.dk}
\acrocite{DTU10}{\acs{DTU} Ocean wide Mean Sea Surface height model, 2010}{Andersen2010}
\acrocite{DW95.1}{Desai-Wahr global ocean model}{Desai1995,Desai1997}
%E
\acro{E2E}{End-to-End}
\acro{ESAC}{Earth Science Advisory Council}
\acro{ECEF}{Earth-Centred, Earth-Fixed\acroextra{ reference frame, see \acs{TRF}}}
\acro{ECF}{Earth Centred Fixed\acroextra{ reference frame, see \acs{TRF}}}
\acro{ECI}{Earth-Centred Inertial\acroextra{ reference frame, see \acs{CRF}}}
\acro{EFRF}{Earth-Fixed Reference Frame\acroextra{ reference frame, see \acs{TRF}}}
\acrourl{ECMWF}{European Centre for Medium-Range Weather Forecasts\acroextra{, \acs{UK}}}{www.ecmwf.int}
\acrocite{EGM2008}{\acs{NGA}'s Earth Gravitational Model 2008}{Pavlis2008}
\acrocite{EGM96}{Joint \acs{NASA} \acs{GSFC} and \acs{NIMA} Earth Gravitational Model 1996}{Lemoine1998}
\acrourl{EGSIEM}{European Gravity Service for Improved Emergency Management\acroextra{, \acs{EU} Horizon 2020}}{www.egsiem.eu}
\acro{EIGEN}{European Improved Gravity model of the Earth by New techniques}
\acrocite{EIGEN-CG03C}{\acs{GFZ}\slash\acs{GRGS} \acs{EIGEN}, version 3}{Forste2005}
\acrocite{EIGEN-GL04C}{\acs{GFZ}\slash\acs{GRGS} \acs{EIGEN}, version 4}{Forste2008}
\acrocite{EIGEN-GRACE02S}{\acs{GFZ}\slash\acs{GRGS} \acs{GRACE}-only \acs{EIGEN}, version 2}{Reigber2005}
\acrocite{EIGEN-5C}{\acs{GFZ}\slash\acs{GRGS} \acs{EIGEN}, version 5}{Forste2008a}
\acrocite{EIGEN-6C}{\acs{GFZ}\slash\acs{GRGS} \acs{EIGEN}, version 6}{Shako2014}
\acrocite{EIGEN-6C2}{\acs{GFZ}\slash\acs{GRGS} \acs{EIGEN}, version 6.2}{Forste2012}
\acrocite{EIGEN-6C4}{\acs{GFZ}\slash\acs{GRGS} \acs{EIGEN}, version 6.4}{Forste2014}
\acro{EGG}{Electrostatic Gravity Gradiometer}
\acro{EGG-C}{European \acs{GOCE} Gravity Consortium}
\acrourl{EGU}{European Geophysical Union}{www.egu.eu}
\acro{EKF}{Extended Kalman Filter}
\acrocite{EBA}{Energy Balance Approach}{O'Keefe1957,Jekeli1999}
\acro{EnKF}{Ensemble Kalman Filter}
\acro{ENSS}{European Navigation Satellites Services}
\acro{ENVISAT}{ENVIronmental SATellite} %need reference
\acrocite{EOT08a}{2008 Empirical Ocean Tide model derived from Altimeter data}{Savcenko2008,Savcenko2008a}
\acro{EWH}{Equivalent Water Height}
\acro{EWI}{Elektrotechniek, Wiskunde en Informatica\acroextra{ faculteit (Faculty of Electrical Engineering, Mathematics and Computer Science), \ac{TUD}}}
\acro{EO}{Earth Observation}
\acrourl{EPSRC}{Engineering and Physical Sciences Research Council}{www.epsrc.ac.uk}
\acrourl{ERC}{European Research Council}{erc.europa.eu}
\acrocite{ERS-1}{First European Remote Sensing satellite}{Duchossois1991}
\acrocite{ERS-2}{Second European Remote Sensing satellite}{Francis1995}
\acrocite{ERA-40}{40-years \acs{ECMWF} Re-Analysis}{Uppala2005}
\acrocite{ERA-Interim}{\enquote{Interim} \acs{ECMWF} Re-Analysis}{Dee2011,Berrisford2011}
\acro{ESO}{European Southern Observatory}
\acro{ESRL}{Earth System Research Laboratory\acroextra{ part of \acs{NOOA}}}
\acrourl{ESA}{European Space Agency}{www.esa.int}
\acrourl{ESOC}{European Space Agency}{www.esa.int/About_Us/ESOC}
\acrourl{ESTEC}{European Space Research and Technology Centre}{www.esa.int/About_Us/ESTEC}
\acro{EU}{European Union}
\acrourl{EUMETSAT}{European Organisation for the Exploitation of Meteorological Satellites}{www.eumetsat.int}
%F
\acrocite{F3C}{\acs{FORMOSAT-3}\slash \acs{COSMIC}}{Kuo1999,Kuo2005}
\acrocite{F7C2}{\acs{FORMOSAT-7}\slash \acs{COSMIC-2}}{Ector2010,Cook2013}
\acrourl{FCT}{Funda\c c\~ao para a Ci\^encia e a Tecnologia\acroextra{ (Science and Technology Foundation)}}{www.fct.pt}
\acro{FDDW}{Frequency-Dependent Data Weighting}
\acro{FES}{Finite Element Solution\acroextra{ global tide model}}
\acrocite[ global tide model]{FES94.1}{release 1994 of the \acl{FES}}{LeProvost1994}
\acrocite[ global tide model]{FES95.2}{release 1995 of the \acl{FES}}{LeProvost1998}
\acrocite[ global tide model]{FES99}  {release 1999 of the \acl{FES}}{Lefevre2002}
\acrocite[ global tide model]{FES2004}{release 2004 of the \acl{FES}}{Lyard2006}
\acro{FOC}{Free Of Charge}
\acro{FRT}{Final Report}
\acro{FORMOSAT-3}{Taiwan's Formosa Satellite Mission-3\acroextra{ (see also \acs{F3C})}} %reference added to F3C.cite
\acro{FORMOSAT-7}{Taiwan's Formosa Satellite Mission-7\acroextra{ (see also \acs{F7C2})}} %reference added to F7C2.cite
\acro{FORTRAN}{FORmular TRANslator\acroextra{ programming language}}
%G
\acro{GGSP}{Galileo Geodetic Service Provider}
\acro{GB}{Giga Bytes\acroextra{, \ac{i.e.} $1024^3$ bytes}}
\acro{GCT}{General Conditions of Tender\acroextra{ for \acs{ESA} contracts}}
\acro{GeoQ}{Relativistic geodesy and gravimetry with quantum sensors}
\acro{GEOSAT}{GEOdetic SATellite} %need reference
\acro{GEOSAT-FO}{\acs{GEOSAT} Follow-On} %need reference
\acrourl{GFZ}{German Research Centre for Geosciences \acroextra{, Germany}}{www.gfz-potsdam.de}
\acrocite{GHOST}{\acs{GPS} High precision Orbit determination Software Tool}{Wermuth2010}
\acrocite{GGM02}{\acs{GRACE} Gravity Model 02}{Tapley2005}
\acrocite{GGM03}{\acs{GRACE} Gravity Model 03}{Tapley2007a}
\acrocite{GGM05G}{\acs{GRACE} Gravity Model 05}{Tapley2013}
\acro{GIA}{Glacial Isostatic Adjustment}
\acrocite{GLDAS}{Global Land Data Assimilation System}{Rodell2004}
\acrocite{GloNaSS}{Globalnaya Navigatsionnaya Sputnikovaya Sistema}{Polischuk2004} %need reference
\acro{GNSS}{Global Navigation Satellite System}
\acrocite{GOCE}{Gravity field and steady-state Ocean Circulation Explorer}{Balmino1999,Floberghagen2011} %need reference
\acro{GOCO}{Gravity Observation COmbination}
\acrocite{GOCO01S}{\acs{GOCO} release 01 satellite-only gravity field model}{Pail2010}
\acrocite{GOCO02S}{\acs{GOCO} release 02 satellite-only gravity field model}{Goiginger2011}
\acrocite{GOCO03S}{\acs{GOCO} release 03 satellite-only gravity field model}{Mayer-Gurr2012}
\acrocite{GOCO05S}{\acs{GOCO} release 05 satellite-only gravity field model}{Mayer-Gurr2015}
\acrocite[ model]{GOT}{Goddard Ocean Tide}{Ray1999}
\acro{GPS}{Global Positioning System}
\acrocite{GRACE}{Gravity Recovery And Climate Experiment}{Tapley1996,Tapley2004}
\acrocite{GRACE-FO}{\acs{GRACE} Follow On}{Sheard2012,Larkin2012,Flechtner2014a}
\acrocite[ mission]{GRAIL}{Gravity Recovery and Interior Laboratory}{Lehman2013}
\acrocite{GRAPHIC}{Group and Phase Ionospheric Calibration}{Montenbruck2003a}
\acro{GRAS}{\acs{GNSS} Receiver for Atmospheric Sounding} %need reference
\acro{GRAZIL}{Gravity field of the moon from radio science tracking and inter-satellite measurements of the \acs{GRAIL} spacecraft\acroextra{, \"Osterreichische Forschungsf\"orderungsgesellschaft mbH}}
\acroref{GRF}{Gradiometer Reference Frame}{sec:grf} %if you change this, also change the superscript
\acro{GROOPS}{Gravity Recovery Object Oriented Programming System}
\acrourl{GRS}{Geoscience and Remote Sensing\acroextra{, Faculty of Civil Engineering and Geosciences, \ac{TUD}}}{www.tudelft.nl/en/ceg/over-faculteit/departments/geoscience-remote-sensing/}
\acrocite{GRIB}{GRIdded Binary}{web:GRIB}
\acrourl{GRGS}{Groupe de Recherche de G\'{e}od\'{e}sie Spatiale\acroextra{, France}}{grgs.obs-mip.fr}
\acrourl{GSFC}{Goddard Space Flight Center\acroextra{, \ac{USA}}}{www.nasa.gov/centers/goddard}
\acro{GSOC}{German Space Operations Centre}
\acrourl{GSP}{General Study Programme}{gsp.esa.int}
%H
\acrocite{HASDM}{High Accuracy Satellite Drag Model}{Storz2005}
\acroref{HRF}{Hill Reference Frame}{sec:hrf}
\acro{hlsst}[hl-SST]{High-low Satellite-to-Satellite tracking}
\acro{HPF}{High-Level Processing Facility}
\acro{HPC}{High Performance Computing}
\acro{HTG}{Hypersonic Technology Goettingen\acroextra{ (Hyperschall Technologie G\"ottinge GmbH)}}
\acro{HTTP}{Hypertext Transfer Protocol}
\acro{HWM07}{Horizontal Wind Model 07}
%I
\acro{IAG}{International Association of Geodesy}
\acro{IAPG}{Institute for Astronomical and Physical Geodesy\acroextra{, \acs{TUM}}}
\acro{IAS}{Institute for Advanced Study}
\acro{IAU}{International Astronomical Union}
\acro{ICCT}{Intercommission Committee on Theory}
\acro{ICET}{International Center for Earth Tides}
\acro{ICESat}{Ice, Cloud and land Elevation Satellite} %need reference
\acrourl{ICGEM}{International Centre for Global Earth Models}{icgem.gfz-potsdam.de}
\acro{IERS}{International Earth Rotation Service}
\acro{IF}{Intermediate Frequency}
\acrourl{IfG}{Institute of Geodesy\acroextra{, \acs{TUG}}}{www.itsg.tugraz.at}
\acrocite{IGS}{International \acs{GNSS} Service}{Dow2005}
\acro{InSAR}{Interferometric Synthetic Aperture Radar}
\acro{IP}{Intellectual Property}
\acro{IPCC}{Intergovernmental Panel for Climate Change}
%Iridium NEXT \citep{Gupta2008} %there is no acronym for this, so cannot use \acrocite
\acro{I/Q}{In-phase\slash Quadrature}
\acrocite{IRNSS}{Indian Regional Navigation Satellite System}{Ganeshan2005}
\acro{IRF}{Inertial Reference Frame}
\acro{ISA}{In-situ Accelerations}
\acrourl{ISI}{International Scientific Indexing}{isindexing.com}
\acrourl{ISSI}{International Space Science Institute}{www.issibern.ch}
\acro{ISR}{Inter-Satellite Range}
\acrourl{IST}{Instituto Superior T\'ecnico}{tecnico.ulisboa.pt}
\acro{ITAM}{Institute of Theoretical and Applied Mechanics, \acs{AVCR}}
\acrourl{ITG}{Institut f\"ur Geod\"asie und Geoinformation\acroextra{, Germany}}{www.igg.uni-bonn.de}
\acro{ITSG}{Institute of Theoretical Geodesy and Satellite Geodesy}
\acrocite{ITG-CHAMP01}{\acs{ITG} \acs{CHAMP}-only model, version 1}{Mayer-Gurr2005}
\acrocite{ITG-GRACE2010}{\acs{ITG} \acs{GRACE}-only model, 2010}{Mayer-Gurr2010,Kurtenbach2009}
\acrocite{ITG-GRACE2010s}{\acs{ITG} \acs{GRACE}-only static model, 2010}{Mayer-Gurr2010}
\acrocite{ITG-GRACE02s}{\acs{ITG} \acs{GRACE}-only model, version 2}{Mayer-Gurr2007}
\acrocite{ITG-GRACE03s}{\acs{ITG} \acs{GRACE}-only model, version 3}{Mayer-Gurr2007a}
\acrocite{ITSG-GRACE2014}{\acs{ITSG} \acs{GRACE}-only model, 2014}{Mayer-Gurr2014}
\acrocite{ITSG-GRACE2016}{\acs{ITSG} \acs{GRACE}-only model, 2016}{Klinger2016}
\acrourl{ITRF}{International Terrestrial Reference Frame}{itrf.ensg.ign.fr}
\acro{ITT}{Invitation To Tenders}
\acro{IUGG}{International Union of Geodesy and Geophysics}
%J
\acrocite{Jason-1}{First Jason altimetry mission}{Menard2003}
\acrourl{JAMTEC}{Japan Agency for Marine-Earth Science and Technology}{www.jamstec.go.jp/e/}
\acrourl{JPL}{Jet Propulsion Laboratory\acroextra{, \acs{USA}}}{www.jpl.nasa.gov}
%K
\acrocite{KANTHA2}{Khanta global ocean model}{Kantha1995,Kantha1995a}
\acro{KB}{Kinematic Baseline}
\acro{KBR}{K-Band Ranging}
\acro{KO}{Kinematic Orbit}
\acro{K/O}{Kick Off}
%L
\acro{L0}{Level 0\acroextra{ data}}
\acro{L1}{\unit{1575.42}{mHz} \acs{GPS} carrier\acroextra{ frequency}}
\acro{L1data}[L1]{Level 1\acroextra{ data}}
\acro{L1A}{Level 1A\acroextra{ data}}
\acro{L1B}{Level 1B\acroextra{ data}}
\acro{L2}{\unit{1227.60}{mHz} \acs{GPS} carrier\acroextra{ frequency}}
\acro{L2data}[L2]{Level 2\acroextra{ data}}
\acro{L2PS}{Level 2 Processing System}
\acro{L5}{\unit{1176.45}{mHz} \acs{GPS} carrier\acroextra{ frequency}}
\acrocite[ hydrological model]{LaD}{Land Dynamics}{Milly2002}
\acrourl{LDAS}{Land Data Assimilation Systems}{ldas.gsfc.nasa.gov}
\acro{LAN}{Local Area Network}
\acrocite{LAGEOS}{LAser GEOdynamics Satellite}{Cohen1985}
\acro{LASER}{Light Amplification by the Stimulated Emission of Radiation}
\acrourl{latex}{\LaTeX}{www.latex-project.org}
\acro{LEGOS}{Laboratoire d'Etudes en Geophysique et Oceanographie Spatiale}
\acro{LEO}{Low-Earth Orbit}
\acro{LGM}{Last Glacial Maximum}
\acrourl{Lic.}{Licenciate\acroextra{ (Licenciate in Engineering, pre-Bologna Accords)}}{en.wikipedia.org/wiki/Licentiate_(degree)\#Portugal}
\acrocite{LISA}{Laser Interferometer Space Antenna}{Merkowitz2003}
\acroref{LHRF}{Local Horizontally-aligned Reference Frame}{sec:lhrf} %if you change this, also change the superscript
\acro{llsst}[ll-SST]{low-low Satellite-to-Satellite Tracking}
\acro{LNA}{Low-Noise Amplifier}
\acro{LNOF}{Local North-Oriented Frame}
\acroref{LORF}{Local Orbital Reference Frame}{sec:lorf} %if you change this, also change the superscript
\acro{LoS}{Line of Sight} %if you change this, also change uv.los
\acroref{LoSRF}{Line-of-sight Reference Frame}{sec:losrf} %if you change this, also change the superscript
\acro{LNA}{Low-Noise Amplifier}
\acro{LS}{Least-Squares}
%M
\acrourl{MATLAB}{MATrix LABoratory}{www.mathworks.com}
\acrocite[ approach]{MasCon}{Mass Concentration}{Rowlands2005,Lemoine2007a}
\acro{MB}{Mega Bytes\acroextra{, \ac{i.e.} $1024^2$ bytes}}
\acro{MEMS}{Micro Electro-Mechanical System}
\acrocite{MetOp}{Meteorological Operational satellite programme}{Edwards2000}
\acro{MetOp-A}{first satellite of the \acl{MetOp}} %need reference
\acro{MGEX}{Multi-GNSS Experiment}
\acro{MODK}{Multiple satellites Orbit Determination using Kalman filtering}
\acrocite{MOG2D-G}{2D Gravity Waves model}{Carrere2003}
\acro{MPP}{Milestone Payment Plan}
\acro{MTR}{Mid-Term Review}
\acro{FR}{Final Review}
%N
\acro{N/A}{Not Applicable}
\acro{NASA}{National Aeronautics and Space Administration\acroextra{, \acs{USA}}}
\acro{NAO}{National Astronomical Observatory\acroextra{, Japan}}
\acrocite{NAO.99b}{1999 release of the \acs{NAO}'s global ocean model}{Matsumoto2000}
\acro{NAVSTAR}{NAVigation System with Time And Ranging} %need reference
\acro{NC}{National Currency}
\acro{NCAR}{National Center for Atmospheric Research}
\acrocite{NCEP/NCAR RP}{\acs{NCEP}/\acs{NCAR} Reanalysis Project}{Kalnay1996}
\acrourl{NCEP}{National Centers for Environmental Prediction\acroextra{, \acs{USA}}}{www.ncep.noaa.gov}
\acrocite{NetCDF}{Network Common Data Form}{Unidata2010}
\acrourl{NERC}{Natural Environment Research Council}{www.nerc.ac.uk}
\acrourl{NGA}{National Geospatial-Intelligence Agency\acroextra{, \acs{USA} (previously called \acs{NIMA})}}{www.nga.mil}
\acro{NGO}{Non-Governmental Organisations}
\acro{NIMA}{National Imagery and Mapping Agency}
\acro{NLDAS}{North American Land Data Assimilation System}
\acro{NOOA}{National Oceanic and Atmospheric Administration}
\acrocite{NRLMSISE}{\acroextra(US )Naval Research Laboratory Mass Spectrometer and Incoherent Scatter Radar\acroextra{ tmospheric model}}{Picone2002}
\acro{NRTDM}{Near Real-Time Density Model}
\acrourl{NTP}{Network Time Protocol}{www.ntp.org}
%O
\acrourl{OCA}{Observatoire de la C\^{o}te d'Azur}{www.oca.eu}
\acrocite{OMCT}{Ocean Model for Circulation and Tides}{Thomas2001}
\acro{ORD}{Open Research Data}
\acro{ORI}{Ocean Research Institute\acroextra{, University of Tokyo}}
\acrocite{ORI-NAO96}{1996 release of the \acs{ORI}-\acs{NAO}'s global ocean model}{Matsumoto1995}
\acro{OS}{Operating System}
\acrourl{OSU}{Ohio State University}{www.osu.edu}
%P
\acro{PB}{Peta Bytes\acroextra{, \ac{i.e.} $1024^5$ bytes}}
\acro{P-code}{Precision code}
\acrocite[ software]{PANDA}{Position And Navigation Data Analyst}{Zhao2004}
\acro{PC}{Personal Computer}
\acro{PCA}{Principal Component Analysis}
\acrocite[ method]{PCCG}{Pre-Conditioned Conjugate Gradient}{Hestenes1952}
\acro{PCDP}{Personal Career Development Plan}
\acrourl{PDF}{Portable Data Format}{en.wikipedia.org/wiki/Portable_Document_Format}
\acro{PDGS}{Payload Data Ground Segment}
\acro{PDO}{Purely Dynamic Orbit}
\acrourl{PECS}{Programme for European Cooperating States}{www.esa.int/About_Us/Plan_for_European_Cooperating_States}
\acro{PI}{Principal Investigator}
\acro{PM}{Progress Meeting}
\acro{POD}{Precise Orbit Determination}
\acrocite{PO.DAAC}{Physical Oceanography Distributed Active Archive Center}{PODAAC2011}
\acrocite{PPHA}{Pacanowski, Ponte, Hirose and Ali barotropic ocean model}{Hirose2001,Ali2003}
\acro{ppm}{parts per million}
\acro{PPP}{Precise Point Positioning}
\acro{PPS}{Precise Positioning Service}
\acro{pps}{pulse per second}
\acro{PRN}{Pseudo-Random Noise}
\acro{PSD}{Power Spectral Density\acroextra{, equal to the square of the \acs{ASD}}}
\acro{PSK}{Phase-Shift Keying}
\acro{PSO}{Precise or Post-processed Science Orbit}
\acro{PSS}{Procedures Specifications and Standards}
%Q
\acro{QWG}{Quality Working Group}
\acrocite{QZSS}{Quasi Zenith Satellite System}{Inaba2009}
%R
\acrourl{RADS}{Radar Altimeter Database System}{rads.tudelft.nl}
\acro{RAM}{Random Access Memory}
\acro{RDO}{Reduced Dynamic Orbit}
\acro{REF}{Research Excellence Framework}
\acrourl{REF2014}{\acl{REF} 2014}{www.ref.ac.uk}
\acro{RF}{Radio Frequency}
\acro{RINEX}{Receiver Independent Exchange\acroextra{ Format}}
\acrocite{RossInv2002}{Ross Sea Height-Based Tidal Inverse Model}{Padman2010b}
\acro{RMS}{Root Mean Squared}
\acro{RRC}{Residual Range Combinations}
\acrocite{RSC94}{\acs{GSFC} Ray-Sanchez-Cartwright global ocean model}{Cartwright1991}
%S
\acro{SA}{Selective Availability}
\acrocite{SAA}{Short-Arcs Approach}{Mayer-Gurr2006a}
\acro{SAR}{Synthetic Aperture Radar}
\acro{S/C}{spacecraft}
\acrocite{SCARF}{Swarm Satellite Constellation Application and Research Facility}{Olsen2013a}
\acro{SCP}{Secure Copy}
\acro{SCoT}{Special Conditions of Tender\acroextra{, \href{http://www.space.dtu.dk/english/-/media/Institutter/Space/forskning/projekter/swarm/SwarmDISC/SD-ITT-1_1/SW-TC-DTU-GS-111_ITT1-1_Special_Conditions_of_Tender.ashx?la=da}{Doc. Ref. SW-TC-DTU-GS-111\_ITT1-1}}}
\acro{SD}{Single-Differenced}
\acro{SD-E}{\acl{SD} phase measurements between Epochs}
\acro{SD-S}{\acl{SD} phase measurements between \acs{GPS} Satellites}
\acro{SDR}{Software Defined Radio}
\acrourl{SES}{School of Earth Science\acroextra{, \acs{OSU}}}{earthsciences.osu.edu}
\acro{SFTP}{Secure File Transfer Protocol}
\acro{SGG}{Satellite Gravity Gradient}
\acro{SG}{Satellite Gradiometry}
\acrocite{SLR}{Satellite Laser Ranging}{Smith1993,Combrinck2010}
\acro{S-LR}[SLR]{Sea-Level Rise}
\acro{SoW}{Statement of Work\acroextra{, \href{http://www.space.dtu.dk/english/-/media/Institutter/Space/forskning/projekter/swarm/SwarmDISC/SD-ITT-1_1/SW-SW-DTU-GS-111_ITT1-1_SoW.ashx?la=da}{Doc. Ref. SW-SW-DTU-GS-111\_ITT1-1}}}
\acrourl{SP3c}{Extended Standard Product 3 Orbit Format}{igscb.jpl.nasa.gov/igscb/data/format/sp3c.txt}
\acro{SPOT}{Satellite Pour l'Observation de la Terre}
\acro{SPS}{Standard Positioning Service}
\acro{SNR}{Signal-to-Noise Ratio}
\acrocite{SR95.1}{\acs{TUD}/\acs{GSFC} Schrama-Ray global ocean models}{Schrama1994}
\acroref{SRF}{Satellite Reference Frame}{sec:srf} %if you change this, also change the superscript
\acro{SRTM}{Shuttle Radar Topographic Mission} %need reference
\acro{SSAC}{Space Science Advisory Council}
\acrourl{SSAU}{Samara State Aerospace University}{www.ssau.ru/english}
\acro{SSE}{Space Systems Engineering\acroextra{, Faculty of Aerospace Engineering, \ac{TUD}}}
\acro{SSO}{Sun-Synchronous Orbit}
\acro{SST}{Satellite-to-Satellite tracking}
\acrocite{SST-AUX-2}{\acs{GOCE} \acs{HPF} non-tidal dealiasing product}{Gruber2014b}
\acro{STD}{STandard Deviation}
\acrocite[satellites ]{Swarm}{Earth's Magnetic Field and Environment Explorers}{Friis-Christensen2006}
\acro{STSE}{Support to Science Element}
\acro{STR}{Star TRacker}
\acrourl{SVN}{Subversion}{subversion.apache.org}
\acrocite{SCW80}{Schwiderski global tide model}{Schwiderski1980,Schwiderski1980a}
\acro{Swarm-ESL}{Swarm Expert Science Laboratories}
%T
%\acro{TA}{True Anomaly}
\acro{TA}{Teaching Assistant}
\acro{TB}{Tera Bytes\acroextra{, \ac{i.e.} $1024^4$ bytes}}
\acro{TBA}{To Be Announced}
\acro{TBD}{To Be Determined}
\acro{TC}{Trailing\slash Cartwheel}
\acro{TD}{Triple-differenced}
\acro{TDP}{Technical Data Package}
\acrourl{TDRSS}{Tracking and Data Relay Satellite System}{tdrs.gsfc.nasa.gov}
\acro{TDW}{Thermospheric Density and Wind}
\acrourl{TLE}{Two-Line Elements}{www.space-track.org}
\acro{TOPEX}{TOPography EXperiment}
\acrocite{TOPEX/Poseidon}{\acs{TOPEX}\slash Poseidon}{Stewart1981}
\acrocite{TPXO}{ToPeX Ocean tidal model}{Egbert1994,Egbert2002}
\acroref{TRF}{Terrestrial Reference Frame}{sec:trf} %if you change this, also change the superscript
\acro{TS}{Trailing\slash Screw}
\acro{TT}{Terrestrial Time}
\acrourlc{TU Delft}{TUD}{Delft University of Technology}{www.tudelft.nl}
\acrourl{TUDAT}{\acs{TUD} Astrodynamics Toolbox}{tudat.tudelft.nl}
\acrourl{TUG}{Graz University of Technology\acroextra{, Austria}}{www.tugraz.at}
\acrocite[  gravity field model]{TUG-CHAMP04}{\acs{TUG} \acs{CHAMP} 2004}{Badura2006}
\acrourl{TUM}{Technische Universit\"at M\"unchen\acroextra{, Germany}}{www.tum.de}
\acrocite[ gravity field model]{TUM-2Sp}{\acs{TUM} 2Sp}{Foldvary2004}
\acro{TVGOGO}{Time Variable Gravity Observed by \acs{GPS}-derived Orbit positions\acroextra{, \"Osterreichische Forschungs-f\"orderungsgesellschaft mbH}}
\acro{TWS}{Terrestrial Water Storage}
%U
\acro{UHF}{Ultra High Frequency\acroextra{ band of the \acs{RF} spectrum}}
\acro{UK}{United Kingdom}
\acrourl{UMinho}{Universidade do Minho}{www.uminho.pt/EN}
\acro{UN}{United Nations}
\acro{URL}{Uniform Resource Locator}
\acrourl{UTexas}{University of Texas at Austin}{www.utexas.edu}
\acro{UTC}{Coordinated Universal Time}
\acrourl{ULisboa}{Universidade de Lisboa\acroextra{ University of Lisbon}}{www.ulisboa.pt/en}
\acro{USA}{United States of America}
\acro{USB}{Universal Serial Bus}
%V
\acro{V2}{Version 2\acroextra{ of the \acs{L2PS}}}
\acro{VHF}{Very High Frequency\acroextra{ band of the \acs{RF} spectrum}}
\acrourl{VZLU}{Aeronautical Research and Test Institute\acroextra{ (V\'yzkumn\'y a zku\u sebn\'i leteck\'y \'ustav)}}{www.vzlu.cz/en}
%X
%W
\acro{WAV}{Windows Wave\acroextra{ file format}}
\acro{WBS}{Work Breakdown Structure}
\acrocite{WGHM}{Watergap Global Hydrological Model}{Doll2003,Doll2014a}
\acrocite{WGS84}{World Geodetic System 1984}{Mularie2000}
\acro{WP}{Work Package}
%Y
%Z
\acro{ZD}{Zero-differenced}
\end{acronym}
\acresetall


% ------ state home or work address -------------------------------

\newtoggle{homecontact}
\togglefalse{homecontact}
% \toggletrue{homecontact}
\newcommand{\homecv}[2]{\iftoggle{homecontact}{#1}{#2}}

% ------ CV version: professional or academic -------------------------------

\newtoggle{professionalcv}
\togglefalse{professionalcv}
% \toggletrue{professionalcv}
\newcommand{\procv}[2]{\iftoggle{professionalcv}{#1}{#2}}

% ------ Verbose description of milestones -------------------------------

\newtoggle{verbose}
% \togglefalse{verbose}
\toggletrue{verbose}
\newcommand{\verbcv}[2]{\iftoggle{verbose}{#1}{#2}}


% ------ Customizations --------------------------------------------------------

% % Customize page headers
% \pagestyle{myheadings}
% \markright{\it{\name}}

% Don't indent paragraphs.
\setlength\parindent{0pt}

% ------ Start --------------------------------------------------------

\begin{document}

\centerline
{\huge \rm \textbf \name}

\procv{}{\vspace{0.25in}
\centering
Postdoctoral Fellow, Center for Space Research, University of Texas at Austin
}

% ------- Personal data ---------------------------------------------------

\begin{cvsection}{Personal Information}

{\bf Full Name:} & Jo\~ao~Greg\'orio~de~Teixeira~da~Encarna\c c\~ao \\
{\bf Birth:} & $25^{\sf th}$ of February 1977 at Funchal, Portugal \\
{\bf Nationality:} &  Portuguese\\

\homecv{%
  {\bf Marital Status:} & Single \\
  {\bf Address:}   & 4303 Duval Street 302
                    78751, Austin Texas, USA\\
  {\bf Telephone:} & +1 512 765 1351\\
  {\bf Email:}     & \href{mailto:j_encarnacao@yahoo.com}{j\_encarnacao@yahoo.com}\\
  {\bf Web:}       & \dynhref{nl.linkedin.com/in/joaoencarnacao}{LinkedIn}\\
}{%
  {\bf Address:}   & 3925 W Braker Lane\newline
                     Ste 200 - WPR 2.9076\newline
                     Austin TX 78759-5316, \ac{USA}\\
  {\bf Telephone:} & +1 (512) 232-6897\\
  {\bf Email:}     & \href{mailto:teixeira@csr.utexas.edu}{teixeira@csr.utexas.edu}\\
  {\bf Web:}       & \dynhref{directory.utexas.edu/index.php?q=joao+encarnacao}{University of Texas},
                     \dynhref{nl.linkedin.com/in/joaoencarnacao}{LinkedIn},
                     \dynhref{www.researchgate.net/profile/Joao_Encarnacao2}{ResearchGate},
                     \dynhref{scholar.google.com/citations?user=k2liFwQAAAAJ}{Google Scholar},
                     \dynhref{orcid.org/0000-0001-6824-2733}{ORCID},
                     \dynhref{www.mendeley.com/profiles/joao-encarnacao4/}{Mendeley},
                     \dynhref{www.scopus.com/authid/detail.uri?authorId=15135565900}{SCOPUS},
                     \dynhref{publons.com/a/782170/}{Publons},
                     \dynhref{github.com/jgte}{GitHub}\\
}
\end{cvsection}

% ------- Summary ---------------------------------------------------

% \verbcv{% !TEX root = ./summary_jgte.tex

\begin{cvtext}{Summary}

I am a researcher in the field of satellite geodesy, with a background in Aerospace Engineering.

%PhD expertise
During my PhD, I have worked with different types of satellite gravimetric data, namely \acl{hlsst} (\aclp{KO}), \acl{llsst} (\aclp{ISR}), and \acl{SGG} (differential accelerometer measurements).
My PhD research focused on:
\begin{itemize}[topsep=0pt,itemsep=1pt,parsep=0pt,partopsep=0pt]
\item modelling the data errors accurately and how its amplitude and spectra influences the quality of the resulting gravity field models;
\item quantifying the error budget of future gravimetric satellite missions, to an unprecedented level of detail;
\item analysing several mission concepts and modelled their error budget in terms of the observations and gravity field parameters;
\item establishing that a constellation of numerous non-dedicated satellites make it possible to measure fast mass transport processes;
\item demonstrating that some mission concepts (those with large radial distances, \acs{e.g.} the cartwheel formation) are very sensitive to particular types of errors (specifically errors connected with \ac{GPS} observations); %
\item proved that alternative mission concepts (the cross-track pendulum formation) are much better suited to complement planned future gravimetric missions.%
\end{itemize}
This has allowed me to study future gravimetric missions in detail, even unconventional ones such as augmenting dedicated gravimetric missions with a large constellation of non-dedicated satellites.
My expertise on this topic has been noted by peers, who have invited me to participate in numerous research projects involving international teams.

%before CSR
As a Post-doctoral Fellow at \ac{TUD}, I dedicated my efforts to implement the Level 2 data processing facility of the \dynhref{https://earth.esa.int/web/guest/missions/esa-operational-eo-missions/swarm}{Swarm satellite mission}, concerning the Precise Orbit Determination and Thermospheric Neutral Density processing streams.
I have acquired expertise in \ac{DSP} techniques and contributed to the processing of Swarm accelerometer data, by combining non-gravitational accelerations derived from \ac{GPS} data and the accelerometer measurements.
In doing so, I have greatly removed the long-term bias in the accelerometer data.
During this time, I also matured my skills in data management and automated processing.

%current situation
Currently as a Scientist Associate at \ac{CSR} of the \ac{UTexas}, I am studying ways of exploiting the maximum resolution and accuracy of the measurements collected by the \ac{GRACE} satellites.
My work focuses on:
\begin{itemize}
\item the calibration of the accelerometers, particular relevant after 2011, when the thermal control on the satellites was switched off;
\item testing large number of unconventional parametrization schemes;
\item developing time-series analysis methods and processing suitable gravimetric data to predict the long-term trends in the \ac{GRACE} gravity field models over the \ac{GRACE}\slash\ac{GRACE-FO} gap; and
\item developing novel methods of connecting the L1B data directly to parameters describing hydrological, solid Earth and glaciological models.
\end{itemize}

%cooperation
I have developed a wide and strong network (AT, CH, CZ, DE, NL, PT and US).
I took the lead in coordinating with several European and US institutes the \dynhref{https://www.researchgate.net/project/Multi-approach-gravity-field-models-from-Swarm-GPS-data}{research and promotion} the gravity field models estimated from the \ac{GPS} data gathered by the Swarm satellite mission.
I also successfully applied for funding to the \ac{DISC} consortium, allowing these activities to proceed smoothly.
In an effort to promote the use of nano-satellites for collecting gravimetric data, I am currently cooperating with \ac{UMinho} and CSR to develop a \ac{MEMS}-based micro accelerometer and investigate practical CubeSat architectures.
The objective of this work is to demonstrate the feasibility of a CubeSat to vastly increase the spatial and temporal sampling of the collected gravimetric data.

%background expertise
I have studied and worked in numerous areas, including Structural Mechanics, Aerodynamics, Preliminary Vehicle Design, Single Stage to Orbit and Laser Propulsion, which have given me the opportunity to broaden my understanding of Physics.
I am an avid programmer, actively learning new languages and techniques in order to better implement the algorithms and procedures required to develop my research.
I openly share the code I developed in \dynhref{http://github.com/jgte}{GitHub}.

\end{cvtext}}{}

% ------- Education ---------------------------------------------------

\begin{cvsection}{Education}

2015 &
  \begin{itti}
    \item \textbf{PhD in Space Geodesy}
    \item \acf{GRS}, \acf{TUD}
    \item Dissertation: \dynhref{tinyurl.com/SatGrav}{\emph{Next-generation satellite gravimetry for measuring mass transport in the Earth system}}
    %TODO: fix these URLs
    \item Promotor: \dynhref{www.tudelft.nl/en/ceg/over-faculteit/departments/geoscience-remote-sensing/staff/scientific-staff/profdr-ing-habil-r-roland-klees/}{Prof. Dr-Ing. habil. Roland Klees}
    \item Supervisor: \dynhref{www.tudelft.nl/citg/over-faculteit/afdelingen/geoscience-remote-sensing/staff/scientific-staff/dr-pg-pavel-ditmar/}{Dr. Ir. Pavel Ditmar}
  \end{itti}\\


2004 &
  \begin{itti}
    \item \textbf{Master of Sciences in Aerospace Engineering}
    \item \acf{AS}, \ac{TUD}
    \item Final Thesis: \emph{Numerical Simulation of Launch Vehicles}
    \item Supervisor: \dynhref{www.tudelft.nl/en/staff/b.a.c.ambrosius}{Prof. Ir. B.A.C. Ambrosius}
  \end{itti}\\

2000 &
  \begin{itti}
    \item \textbf{\acf{Lic.} in Aerospace Engineering}
    \item \acf{IST}, \acf{UTL}
    \item 5$^{\rm th}$ year concluded at \ac{TUD}, through the \dynhref{www.erasmusprogramme.com}{ERASMUS program}
    \item Report: \emph{Optimum Aerodynamic Shape for a High Altitude Long Endurance Aerostatic Platform}
    \item Supervisor: Prof. Dr. Ir. Theo van Holten
  \end{itti}\\

\end{cvsection}

% \citem{Scholarships}\\
% Socrates/ERASMUS Scholarship of the European Union (September 1999 -- June
% 2000)


% % -------- Work experience --------------------------------------------

\begin{cvsection}{Academic and Research Experience}

Aug. 2016 -- present &
  \begin{itti}
    \item \textbf{Research Associate} at \acf{CSR}, \acf{UTexas}, \ac{USA}:
    \begin{ittib}
      \item Improvements in the calibration of the accelerometers on-board the \acf{GRACE} satellites, in particular in what relates to temperature effects;
      \item Determination of the (non-linear) long-term trends in the \ac{GRACE} gravity field solutions and their prediction during the \ac{GRACE}\slash\acf{GRACE-FO} gap;
      \item Time-varying gravity fields estimated from Kinematic Orbits;
      \item In-house software development in Matlab and Ruby.
    \end{ittib}
  \end{itti}\\

Sep. 2011 -- Jul. 2016 &
  \begin{itti}
    \item \textbf{Research Associate} at \ac{AS}, \ac{TUD}, the Netherlands:
    \begin{ittib}
      \item Calibration of the accelerometers on-board the Swarm satellites;
      \item Improvements in the modelling of non-conservative forces acting on satellites;
      \item Exploiting \acf{DSP} techniques to merge the measurement of non-gravitational accelerations from different sources: \acf{GPS}-driven and accelerometer observations;
      \item Time-varying gravity fields estimated from Kinematic Orbits;
      \item Research project: Assessment of Satellite Constellations for Monitoring the Variations in Earth's Gravity Field;
      \item Research project: GOCE+ Theme3: Air density and wind retrieval using \acf{GOCE} data;
      \item Research project: Development of the Swarm Level 2 Algorithms and Associated Level 2 Processing Facility;
      \item In-house software development in Fortran, Matlab and Ruby;
      \item Student supervision and mentoring.
    \end{ittib}
  \end{itti}\\

Jan. 2007 -- Dec. 2015 &
  \begin{itti}
    \item \textbf{PhD Candidate} at \ac{GRS}, \ac{TUD}, the Netherlands:
    \begin{ittib}
      \item Simulation of future gravimetric satellite missions and noise budget of low-low satellite-to-satellite tracking gravimetric data;
      \item Impact of orbit position modelling errors in the quality of satellite gravimetric data;
      \item Retrieval of the high-frequency time-variable gravity field of the Earth with numerous satellites;
      \item Research project: Assessment of a Next Generation Gravity Mission for Monitoring the Variations of Earth's Gravity Field;
      \item Research project: Monitoring and Modelling Individual Sources of Mass Distribution and Transport in the Earth System by Means of Satellites;
      \item In-house software development in Fortran and Matlab;
      \item Student supervision and mentoring.
    \end{ittib}
  \end{itti}\\


Apr. 2005 -- Nov. 2006 &
  \begin{itti}
    \item \textbf{Stress Engineer} at \dynhref{www.globaltechnics.nl/}{Global Technics}, Leiden, the Netherlands:
    \begin{ittib}
      \item Automated design (for weight and stress minimization) of fuselage panels for the Airbus A380 aircraft (in-house implementation of a tool in C++);
      \item Trainees supervision and mentoring.
    \end{ittib}
  \end{itti}\\

\procv{
6/2005 -- 7/2005 &
  \begin{itti}
    \item \textbf{Systems Engineer} at \dynhref{www.reduct.net}{Reduct}, Kontich, Belgium
  \end{itti}\\
}{}

Oct. 2004 -- Jan. 2005 &
  \begin{itti}
    \item \textbf{Aerospace Engineer} at \dynhref{www.delta-utec.com/}{Delta-Utec}, Leiden, the Netherlands:
    \begin{ittib}
      \item Contractor Work: Implementation of a Sub-Orbital Optimization Module into the Simulation Tool \emph{COLVET} (developed in-house at TU Delft).
    \end{ittib}
  \end{itti}\\

Mar. 2004 -- Apr. 2004 &
  \begin{itti}
    \item \textbf{Trainee} at the Prins Maurits Laboratorium, \dynhref{www.tno.nl/}{TNO}, the Netherlands
    \item Supervisor: Ir. Berry Sanders, Rocket Technology Research Group:
    \begin{ittib}
      \item Implementation of the Launch Vehicle Simulation and Optimisation Tool \emph{COLVET};
      \item Numerical Simulations on Laser Propulsion (appendix of MSc thesis);
      \item Collaboration with international colleagues (PT and NL) on a \acf{ESA}-funded project to determine the feasibility of Laser Propulsion.
    \end{ittib}
  \end{itti}\\

Sep. 2001 -- Dec. 2001 &
  \begin{itti}
    \item \textbf{Trainee} at \acf{ESTEC}, \ac{ESA}, Noordwijk, the Netherlands
    \item Supervisor: \dynhref{en.wikipedia.org/wiki/Wubbo_Ockels}{Prof. Wubbo Ockels}:
    \begin{ittib}
      \item Collaboration with fellow MSc colleagues on a space mission design project: \emph{Lunar Exploration with Ariane 5};
      \item Simulation of rocket ascent trajectories (implemented a 2D orbit integrator in Matlab);
      \item Optimization or rocket trajectories, thrust and attitude program, fuel consumption and payload;
      \item Preliminary lunar mission design.
    \end{ittib}
  \end{itti}\\

\end{cvsection}

\verbcv{
\begin{cvtext}{Teaching Experience}

At TU Delft, I was required to supervise student projects of a practical nature, every year.
This means that I had to direct the work of a small group of students (7 to 9) to a particular objective.
It is an activity I enjoy doing and I see the students are enthusiastic about.
I always ask them to fill (anonymously) a short list of questions regarding their opinion of the project and my ability as instructor (these answer sheets are available if requested).
I always get encouraging and positive feedback.

My senior colleagues always grade my teaching activities as very good to excellent and I am often asked by students to support their application with recommendation letters.

In what concerns teaching large groups of students, I have given lectures to classes of about 30 students, on exceptional occasions, at the request of colleagues.

Additionally, I was a lecturer at the 2017 Summer School On Data Assimilation And Its Applications In Oceanography, Hydrology, Risk \& Safety And Reservoir Engineering (cf. \dynhref{data-assimilation.com}{data-assimilation.com}).

I value all teaching my experiences because they were extremely rewarding.

\end{cvtext}
}{}

% \iftoggle{expliciturl}{\pagebreak}{}


% % ------- Skills ------------------------------------------------------

\begin{cvsection}{Skills}

Communication: & Numerous presentations of research results (8 oral and 4 poster)\\

Teaching: & \vspace{-1.5ex}
  \begin{ittib}
    \item Student supervision in the context of individual and group assignments
    \item Introductory lectures to the practical projects
  \end{ittib}\\

Theoretical: & \vspace{-1.5ex}
  \begin{ittib}
   \item Parametric inversion
   \item Statistical analysis
   \item Stochastic modelling
   \item Spherical harmonic functions
   \item Digital signal processing
   \item Coordinate transformations\slash quaternion arithmetic
   \item Fourier analysis
 \end{ittib}\\


\procv{}{
Articles review: & Successfully completed the review of 8 scientific articles, cf. \dynhref{publons.com/a/782170/}{Publons}\\
}

Computational: & \vspace{-1.5ex}
  \begin{ittib}
    \item Algorithm development and implementation
    \item Data management, analysis and visualisation
    \item Automation, robustness, fault recovery
    \item Problem resolution\slash solution discovery\slash hacking
 \end{ittib}\\

Software: & Latex, MS Office, Git, SVN\\

Programming: & \vspace{-1.5ex}
  \begin{ittib}
    \item 1996 -- present: Bash
    \item 1998 -- present: Matlab
    \item 2002 -- present: Fortran
    \item 2006 -- 2008: C/C++
    \item 2011 -- present: Ruby
    \item 2015 -- present: Python
  \end{ittib}\\

\acp{OS}: & OSX, MS Windows, Unix/Linux \\

\end{cvsection}

% % ------- Fields of interest ------------------------------------------

\begin{cvsection}{Fields of Interest}
 Space geodesy & \\
 Earth System Science & \\
 Mathematical Modelling & \\
 Digital signal processing & \\
 Numerical Simulation & \\
 Big data & \\
 Rocket Motion and Orbital Mechanics & \\
 System Analysis and Design & \\
 Aerodynamics & \\
 Structural Mechanics & \\
\end{cvsection}


% % ------- Collaborations ------------------------------------------------------

\procv{}{
\begin{cvsection}{Collaborations}

2017 -- present &
Collaboration with Dr. Guillaume Ramillien from \acf{CNRS} and Dr. Ale\v{s} Bezd\v{e}k the \acf{ASU}  of the \acf{AVCR} to {\bf drive surface mass variations directly from \enquote{reduced} gravimetric data } (i.e. observations \enquote{cleaned} of non-gravitational and trivial gravitational effects).\\

2017 -- present &
Collaboration with Dr. Noble Hatten and Dr. Dae Lee of the \ac{CSR}. \ac{UTexas} for the {\bf development of a CubeSat architecture that replicates the gravimetric capabilities of the \ac{GRACE} satellites}.\\

2016 -- present &
International collaboration with Prof. Luis Rocha of \acf{UMinho} and Dr. Dae Lee of the \ac{CSR}, \ac{UTexas} for the {\bf development of a \ac{MEMS}-based space accelerometer as a first step towards the nano-gravimetric satellite framework}.\\

2015 -- present &
International collaboration with
Prof. Torsten Mayer-G\"{u}rr of the \acf{IfG} of the \acf{TUG},
Dr. Ale\v{s} Bezd\v{e}k of the \ac{ASU}  of the \acf{AVCR},
Prof. Adrian J\"{a}ggi of the \acf{AIUB},
Prof. Pieter Visser of the \dynhref{www.lr.tudelft.nl}{Aerospace Faculty} of the \ac{TUD} and
Prof. C.K. Shum of the \acf{SES} of the \acf{OSU}
for the {\bf study of the time-variable gravity field of the Earth estimated from GPS data collected by the \dynhref{earth.esa.int/web/guest/missions/esa-operational-eo-missions/swarm}{Swarm Satellite mission}}. Within the scope of this project, we submitted a grant application with very positive reviews (Ref. ESA AO/1-7927/14/NL/MP), and have recently been awarded funding under the \dynhref{tinyurl.com/SwarmGrav}{ITT posted by the \ac{ESA}-funded a\ac{DISC} consortium}\\

2014 -- present &
Collaboration with \ac{TUD} on the \dynhref{doptrack.tudelft.nl}{DopTrack project}, consisting of a {\bf satellite tracking radio station that exploits the Doppler effect}; co-initiated and promoted the project, secured departmental funding, selected and assembled the hardware, developed software, engaged students and mentored practical undergraduate projects.\\

\end{cvsection}
}

% % -------- Research Projects ----------------------------------------------------

\procv{}{
\begin{cvsection}{Research Projects}
2013 -- 2015  & Assessment of Satellite Constellations for Monitoring the Variations in Earth's Gravity Field (ESA contract 4000108663/13/NL/MV) \\
2013          & GOCE+ Theme3: Air density and wind retrieval using GOCE data (ESA contract 400010284/11/NL/EL)\\
2011 -- 2016  & Development of the Swarm Level 2 Algorithms and Associated Level 2 Processing Facility (ESA Contract 4000102140/10/NL/JA)\\
2010          & Assessment of a Next Generation Gravity Mission for Monitoring the Variations of Earth's Gravity Field (ESTEC contract 22643/09/NL/AF)\\
2008          & Monitoring and Modelling Individual Sources of Mass Distribution and Transport in the Earth System by Means of Satellites (ESA contract 20403) \\
\end{cvsection}
}

% % -------- Publications ----------------------------------------------------

\procv{}{% !TEX root = ./publications_jgte.tex

% --------------------------------------------------------

\cvheading{Journal publications}
\begin{refsection}
\nocite{
TeixeiraEncarnacao2016,
Bezdek2016,
Siemes2016,
VandenIJssel2015,
HashemiFarahani2013,
Olsen2013a,
Visser2013,
Ditmar2012,
Gunter2011,
Resendes2007}
\togglefalse{bbx:url}
\printbibliography
\end{refsection}

%--------------------------------------

\cvheading{Conference proceedings (peer-reviewed)}
\begin{refsection}
\renewbibmacro*{doi+eprint+url}{%
  \iftoggle{bbx:doi}
    {\printfield{doi}}
    {}%
  \newunit\newblock
  \iftoggle{bbx:eprint}
    {\usebibmacro{eprint}}
    {}%
  \newunit\newblock
  \iftoggle{bbx:url}
    {\iffieldundef{doi}{\usebibmacro{url+urldate}}{}}
    {}%
}
\nocite{
Gunter2012a,
Gunter2010,
Gunter2009,
Encarnacao2008a,
Encarnacao2008,
Resendes2006,
Resendes2005}
\printbibliography
\end{refsection}

%--------------------------------------

\cvheading{Invited Presentations}
\begin{refsection}
\nocite{
TeixeiraEncarnacao2015a, % First monthly gravity field solutions derived from GPS orbits of       Swarm AGU 2015 oral   1
TeixeiraEncarnacao2017b} % Satellite Gravimetry                                                   DASS 2017      oral   2
\printbibliography
\end{refsection}

%--------------------------------------

\cvheading{Conference Attendance}
\begin{refsection}
\nocite{
Encarnacao2002,          % Single Stage To Orbit Minimum Requirements Through Numerical Simulation COSPAR 2002   oral   3 
TeixeiraEncarnacao2007,  % Temporal aliasing in GRACE monthly solutions                            INTERGEO 2007 poster    1
TeixeiraEncarnacao2007a, % Influence of hydrology-related temporal aliasing on the quality of      VMSG 2007     poster    2
Encarnacao2008,          % Analysis of Satellite Formations in the Context of Gravity Field        3rd ISFFMT    oral   4  
TeixeiraEncarnacao2008,  % Spectral analysis of positioning modelling errors in gravimetric data   IAG 2008      poster    3
TeixeiraEncarnacao2014,  % POD-assisted calibration of Swarms Accelerometer Data                   4th SwarmDQW  oral   5
TeixeiraEncarnacao2014a, % Combination of Swarm's Uncalibrated Accelerometer Data with POD-Based   3rd SwarmSM   oral   6
TeixeiraEncarnacao2014b, % Preliminary analysis of accelerometer data                              2nd SwarmDQW  oral   7
Encarnacao2015,          % Impact of Orbit Position Errors on Future Satellite Gravity Models      AGU 2015      poster    4
TeixeiraEncarnacao2015b, % Frequency domain combination of POD-driven and measured accelerations   5th SwarmDQW  oral   8
TeixeiraEncarnacao2016b, % Gravity field models derived from Swarm GPS data                        EGU 2016      poster    5
TeixeiraEncarnacao2017a, % Gravity field models derived from Swarm GPS data                        EGU 2017      poster    6
Encarnacao2017}          % Temperature corrected-calibration of GRACE’s accelerometer              AGU 2017      poster    7
\printbibliography
\end{refsection}

\cvheading{Conference Contributions}
\begin{refsection}
\nocite{
TeixeiraEncarnacao2017, %Swarm as an Observing Platform for Large Surface Mass Transport Processes Banff 2017
gunter2010using,
ditmar2010mitigation,
hashemi2010contribution,
gunter2011investigation,
gunter2012potential,
doornbos2012thermospheric,
Olsen2013,
doornbos2013air,
bruinsma2014results,
iran2014search,
astafyeva2015ionospheric,
doornbos2015processing,
Siemes2015,
jaggi2016european,
sneew2016esa,
siemes2016improvements,
doornbos2016thermospheric}
\printbibliography
\end{refsection}

%--------------------------------------

\cvheading{Miscellaneous Contributions}
\begin{refsection}
\nocite{
Iran-Pour2015,
Anselmi2010}
\printbibliography
\end{refsection}

%--------------------------------------






}

% % ------- Languages ------------------------------------------------------

\begin{cvtext}{Languages}
\begin{tabular}{
l
>{\centering\arraybackslash}m{3cm}
>{\centering\arraybackslash}m{3cm}
>{\centering\arraybackslash}m{3cm}}
   & Speaking & Reading & Writing \\
\hline
Portuguese & \multicolumn{3}{c}{mother tongue}  \\
English\footnote{holder of the \dynhref{www.cambridgeenglish.org/exams/proficiency/index.aspx}{Certificate of Proficiency in English}}    & excellent & excellent & excellent \\
Spanish    & good & good & fair \\
Italian    & good & good & fair \\
Dutch      & fair & fair & limited \\
French     & fair & fair & limited \\
\hline
\end{tabular}
\end{cvtext}

% % ------- Personal development ------------------------------------------------------

\begin{cvsection}{Personal development}

Sep. 2015 & Scientific Writing, Sören Johnson, \ac{TUD}\\

Jul. 2017 & Leading without formal authority, Emil Kresl, \ac{UTexas}\\

Jul. 2017 & Meeting effectiveness, Emil Kresl, \ac{UTexas}\\

Sep. 2017 & Dealing with Difficult People, Jeff Stellmach, \ac{UTexas}\\

\end{cvsection}

% % -------- Other Activities --------------------------------------------

\begin{cvsection}{Sports}
1991 -- 2009 & Basketball \\
April 2006   & Finalist of the \dynhref{www.fortismarathonrotterdam.nl/}{26th International Fortis Marathon of Rotterdam}\\
September 2016 - present & Sailing\\
\end{cvsection}

\begin{cvsection}{Other Activities}
1991 -- 2001 & Scout at the 92$^{nd}$ Scout-group of the \dynhref{www.aep.pt}{Association of Portuguese Escoteiros}\\
1993 -- present & Radio Amateur, call sign CT3IU, class B\\
\procv{1996 -- present & Drivers Licence \\}{}
\end{cvsection}

% % -------- References --------------------------------------------

% \pagebreak

\begin{cvsection}{Referees}
Prof. Byron Tapley &
  \begin{itti}
    \item Research advisor at \ac{CSR} of \ac{UTexas}
    \item +1 512 471 5573
    \item \href{mailto:tapley@csr.utexas.edu}{tapley@csr.utexas.edu}
  \end{itti}\\

Prof. Pieter Visser &
  \begin{itti}
    \item Research advisor at \ac{AS} of \ac{TUD}
    \item +31 15 27 82595
    \item \href{mailto:P.N.A.M.Visser@tudelft.nl}{P.N.A.M.Visser@tudelft.nl}
  \end{itti}\\

Dr. Pavel Ditmar &
  \begin{itti}
    \item PhD advisor at \ac{GRS} of \ac{TUD}
    \item +31 15 27 82501
    \item \href{mailto:p.g.ditmar@tudelft.nl}{p.g.ditmar@tudelft.nl}
  \end{itti}\\

Prof. Boudewijn Ambrosius &
  \begin{itti}
    \item MSc advisor at \ac{AS} of \ac{TUD}
    \item \href{mailto:B.A.C.Ambrosius@tudelft.nl}{B.A.C.Ambrosius@tudelft.nl}
  \end{itti}\\

\end{cvsection}

\vfill
The \dynhref{jgte.github.io/cv/cv_jgte.pdf}{PDF} and \dynhref{jgte.github.io/cv/cv_jgte_print.pdf}{print-ready} versions this document are available on-line.

\label{endpage}
\end{document}
